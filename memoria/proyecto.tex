% !TEX options=--shell-escape
%\documentclass[a4paper,11pt]{book}
\documentclass[a4paper,oneside,11pt,titlepage]{book}
%\usepackage{listings}
\usepackage[utf8]{inputenc}
\usepackage[spanish]{babel}

% \usepackage[style=list, number=none]{glossary} %
\usepackage{titlesec}
\setcounter{secnumdepth}{3}

%\usepackage{pailatino}

\decimalpoint
\usepackage{dcolumn}
\newcolumntype{.}{D{.}{\esperiod}{-1}}
\makeatletter
\addto\shorthandsspanish{\let\esperiod\es@period@code}
\makeatother

\usepackage{booktabs}
%\usepackage[chapter]{algorithm}
\RequirePackage{verbatim}
\usepackage[newfloat]{minted}
\usepackage{caption}
\usepackage{multirow}
%\RequirePackage[Glenn]{fncychap}
\usepackage{fancyhdr}
\usepackage{graphicx}
\usepackage{afterpage}
\usepackage{parskip}
\usepackage{longtable}
\usepackage[normalem]{ulem}
\useunder{\uline}{\ul}{}

\usepackage[pdfborder={000}]{hyperref} %referencia

% ********************************************************************
% Re-usable information
% ********************************************************************
\newcommand{\myTitle}{Desarrollo e Implementación de modelos paralelos de Soft Computing en CUDA\xspace}
\newcommand{\myDegree}{Grado en Ingeniería Informática\xspace}
\newcommand{\myName}{David Criado Ramón\xspace}
\newcommand{\myProf}{Manuel I. Capel Tuñón\xspace}
\newcommand{\myOtherProf}{María del Carmen Pegalajar Jiménez\xspace}
%\newcommand{\mySupervisor}{Put name here\xspace}
\newcommand{\myFaculty}{Escuela Técnica Superior de Ingenierías Informática y de
Telecomunicación\xspace}
\newcommand{\myFacultyShort}{E.T.S. de Ingenierías Informática y de
Telecomunicación\xspace}
\newcommand{\myDepartment}{Departamento de Ciencias de la Computación e Inteligencia Artificial\xspace}
\newcommand{\myUni}{\protect{Universidad de Granada}\xspace}
\newcommand{\myLocation}{Granada\xspace}
\newcommand{\myTime}{\today\xspace}
\newcommand{\myVersion}{Version 0.1\xspace}


\hypersetup{
pdfauthor = {\myName (email (en) ugr (punto) es)},
pdftitle = {\myTitle},
pdfsubject = {},
pdfkeywords = {palabra_clave1, palabra_clave2, palabra_clave3, ...},
pdfcreator = {LaTeX con el paquete ....},
pdfproducer = {pdflatex}
}

%\hyphenation{}


%\usepackage{doxygen/doxygen}
%\usepackage{pdfpages}
\usepackage{url}
\usepackage{colortbl,longtable}
\usepackage{minted}
\usepackage[stable]{footmisc}
\usepackage[table,xcdraw]{xcolor}
\usepackage{longtable}
%\usepackage{index}

%\makeindex
%\usepackage[style=long, cols=2,border=plain,toc=true,number=none]{glossary}
% \makeglossary

% Definición de comandos que me son tiles:
%\renewcommand{\indexname}{Índice alfabético}
%\renewcommand{\glossaryname}{Glosario}

\pagestyle{fancy}
\fancyhf{}
\fancyhead[LO]{\leftmark}
\fancyhead[RE]{\rightmark}
\fancyhead[RO,LE]{\textbf{\thepage}}
\renewcommand{\chaptermark}[1]{\markboth{\textbf{#1}}{}}
\renewcommand{\sectionmark}[1]{\markright{\textbf{\thesection. #1}}}

\setlength{\headheight}{1.5\headheight}

\newcommand{\HRule}{\rule{\linewidth}{0.5mm}}
%Definimos los tipos teorema, ejemplo y definición podremos usar estos tipos
%simplemente poniendo \begin{teorema} \end{teorema} ...
\newtheorem{teorema}{Teorema}[chapter]
\newtheorem{ejemplo}{Ejemplo}[chapter]
\newtheorem{definicion}{Definición}[chapter]
\newenvironment{code}{\captionsetup{type=listing}}{}
\SetupFloatingEnvironment{listing}{name=Código Fuente}
\definecolor{gray97}{gray}{.97}
\definecolor{gray75}{gray}{.75}
\definecolor{gray45}{gray}{.45}
\definecolor{gray30}{gray}{.94}

\usepackage{appendix}


\newcommand{\bigrule}{\titlerule[0.5mm]}


%Para conseguir que en las páginas en blanco no ponga cabecerass
\makeatletter
\def\clearpage{%
  \ifvmode
    \ifnum \@dbltopnum =\m@ne
      \ifdim \pagetotal <\topskip
        \hbox{}
      \fi
    \fi
  \fi
  \newpage
  \thispagestyle{empty}
  \write\m@ne{}
  \vbox{}
  \penalty -\@Mi
}
\makeatother

\usepackage{pdfpages}
\begin{document}
\input{portada/portada}
%!TEX root = ../proyecto.tex
%\thispagestyle{empty}
%\cleardoublepage

%\thispagestyle{empty}

%\input{portada/portada_2}



\cleardoublepage
\thispagestyle{empty}

\begin{center}
{\large\bfseries Desarrollo e Implementación de modelos paralelos de Soft Computing en CUDA.}\\
\end{center}
\begin{center}
David Criado Ramón\\
\end{center}

%\vspace{0.7cm}
\noindent{\textbf{Palabras clave}: Soft Computing, CUDA, GPU, paralelo, mapa auto-organizado, Spark, Kohonen, árbol de decisión, CART, CUDT, Python, Numba, SUSY, reducción, scan.}\\

\vspace{0.7cm}
\noindent{\textbf{Resumen}}\\

El objetivo de este proyecto es la paralelización de varios algoritmos de \textit{Soft Computing} usando la tecnología propietaria de \textit{NVIDIA} para sus tarjetas gráficas, \textit{CUDA}. Dos algoritmos fueron seleccionados para su desarrollo: el mapa auto-organizado de \textit{Kohonen} y el árbol de decisión \textit{CART}. Además, para simplificar el proceso de desarrollo nos apoyamos en el \textit{framework} para computación en clúster \textit{Apache Spark}, que nos ha permitido desarrollar una solución viable para ser utilizada bien en una única máquina con un dispositivo \textit{CUDA} o en un clúster con múltiples máquinas, cada una con su dispositivo CUDA correspondiente.\\

En el caso del mapa auto-organizado de Kohonen planteamos una solución basada en la primitiva paralela de la reducción y el uso de operaciones atómicas. Además, para evitar \textit{overheads} debidos al lanzamiento de múltiples \textit{kernels}, limitamos el tamaño del mapa de salida a 1024 neuronas. \\

En el caso del árbol de decisión \textit{CART}, seguimos una alternativa similar a CUDT, basándonos en la primitiva paralela de suma acumulada o \textit{scan} y limitando los problemas a resolver a clasificación binaria, añadiendo algunas técnicas extra, como los \textit{streams}, para obtener resultados ligeramente mejores en tiempo de ejecución. \\

Para comprobar el rendimiento de ambos modelos hemos utilizado la base de datos \textit{SUSY}, compuesta por 5 millones de muestras con 18 características y una clase binaria correspondiente y, haciendo uso de \textit{NumPy}, hemos desarrollado las versiones equivalentes utilizando la CPU. En las pruebas realizadas llegamos a obtener un tiempo de ejecución hasta 28 más rápido para el mapa auto-organizado de \textit{Kohonen} con el dispositivo \textit{CUDA}. Sin embargo, para el árbol de decisión desarrollado, aunque conseguimos tiempos de ejecución más rápidos, obtenemos resultados mucho más moderados, con tiempos de ejecución hasta 1,5 veces más rápidos.


\thispagestyle{empty}


\begin{center}
{\large\bfseries Development and Implementation of Soft Computing parallel models using CUDA.}\\
\end{center}
\begin{center}
David Criado Ramón\\
\end{center}

%\vspace{0.7cm}
\noindent{\textbf{Keywords}: Soft Computing, CUDA, GPU, parallel, self-organizing map, Spark, Kohonen, decision tree, CART, CUDT, Python, Numba, SUSY, reduction, scan.}\\

\vspace{0.7cm}
\noindent{\textbf{Abstract}}\\

In this project, we parallelize two Soft Computing models using CUDA: self-organizing maps, a competitive and unsupervised learning neuronal network and CART decision trees, an exhaustive search decision tree algorithm suitable for classification and regression tasks (though we are limiting here our implementation, based on GPU, to binary classification problems).\\

Both models are developed using Python and \textit{Apache Spark}. \textit{Python} allows us to implement our code faster via \textit{Numba} than using the more traditional approach with\textit{C++} and \textit{CUDA} and, thereby, it offers an easy integration with Spark. On the other hand, \textit{Spark} offers an easy way to read CSV files, manageable parameters to solve scalability issues (executors, number of cores, étc.) and the capability to carry out only one implementation, which is  suitable for execution either on one machine. In order to compare GPU performance vs CPU, we use NumPY and Spark to create CPU-based implementations that work like the GPU ones.\\

The self-organizing map (SOM), proposed by Kohonen in the early 80s, is an unsupervised learning algorithm suitable for clustering and dimensionality reduction, among others. Our solution, based on the batch version of the SOM, uses Spark's \textit{mapPartition} transformation in order to distribute work among all the partitions of the RDD. Each iteration, on each partition, our implementation uses the Euclidean distance to calculate distance between the neurons, as well as the parallel reduction primitive in order to find the closest neuron (BMU) and updates the weights structure using atomic operations. Furthermore, we limit the number of neurons on the output map to 1024 in order to avoid overheads by multiple kernel launches. We achieve an speedup of almost 28 times compared with the CPU implementation on the selected dataset, SUSY, composed by 5 million instances with 18 features.\\

The development of a decision tree learning algorithm proved to be much more challenging. Being based on CART, we decided to develop a solution similar to CUDT, some small tweaks like common on-line pruning techniques: changing depth and number of samples assigned to the leaves of trees and the use of streams to allow concurrent evaluation of nodes in the GPU. In this algorithm implementation, by using Python limits the implementation of the algorithm since Numba does not support CUDA dynamic
parallelism, thenforcing us to launch multiple kernels instead. Also, being based on CUDT, our implementation only solves binary classification problems and uses the parallel scan primitive, which we implemented using warp constraint. We incorporate Spark to this model by creating a random forest of these trees. This will help to prevent overfitting and will be, in fact, faster than using Spark to create a single tree,	 since the latter would create a serious communication overhead between the RDD's partitions. To test our implementation, we use SUSY dataset (5 million samples, 18 features, 1 binary class label). Our GPU implementation was able to improve the CPU one, however, the best speedup we were able to obtain in our system was only 1.5 times faster than the second one, although parameters like the maximum depth, the number of trees in the random forest or the number of samples of the dataset may have heavily impacted the results obtained.

\newpage
\thispagestyle{empty}

\noindent\rule[-1ex]{\textwidth}{2pt}\\[4.5ex]

Yo, \textbf{David Criado Ramón}, alumno de la titulación Grado en Ingeniería Informática de la \textbf{Escuela Técnica Superior
de Ingenierías Informática y de Telecomunicación de la Universidad de Granada}, con DNI 26254133-R, autorizo la
ubicación de la siguiente copia de mi Trabajo Fin de Grado en la biblioteca del centro para que pueda ser
consultada por las personas que lo deseen.

\vspace{6cm}

\noindent Fdo: David Criado Ramón

\vspace{2cm}

\begin{flushright}
Granada a 3 de septiembre de 2019.
\end{flushright}


\chapter*{}
\thispagestyle{empty}
\noindent\rule[-1ex]{\textwidth}{1pt}\\[4.5ex]
D. \textbf{Manuel Capel Tuñón}, Profesor del Departamento de Lenguajes y Sistemas Informáticos de la Universidad de Granada.

\vspace{0.5cm}

D. \textbf{María del Carmen Pegalajar Jiménez}, Profesora del Departamento de Ciencias de la Computación e Inteligencia Artificial de la Universidad de Granada.


\vspace{0.5cm}

\textbf{Informan:}

\vspace{0.5cm}

Que el presente trabajo, titulado \textit{\textbf{Desarrollo e Implementación de modelos paralelos de Soft Computing en CUDA}},
ha sido realizado bajo su supervisión por \textbf{David Criado Ramón}, y autorizamos la defensa de dicho trabajo ante el tribunal
que corresponda.

\vspace{0.5cm}

Y para que conste, expiden y firman el presente informe en Granada a 3 de septiembre de 2019.

\vspace{1cm}

\textbf{Los directores:}

\vspace{4cm}

\noindent \textbf{Manuel I. Capel Tuñón \ \ \ \ \ María del Carmen Pegalajar Jiménez}

\chapter*{Agradecimientos}
\thispagestyle{empty}

       \vspace{1cm}


A Rubén, por estar siempre apoyándome.  
\frontmatter
\tableofcontents
\listoffigures
\listoftables

%
\mainmatter
\setlength{\parskip}{5pt}
\chapter{Introducción y motivación.}
\section{Motivación.}
La tecnología propietaria \textit{CUDA (Computer Unified Device Architecture)} \cite{cuda} de NVIDIA, presentada en junio de 2007 y aplicable tanto a la arquitectura de las tarjetas gráficas de la misma marca como al modelo de programación genérico asociado, a lo largo de la última década ha supuesto un gran cambio en las implementaciones paralelas de algoritmos y, además, es muy utilizada y popular entre la comunidad científica.\\

La estructura de la GPU, utilizando un mayor número de núcleos a cambio de una velocidad de reloj más baja a la que podemos encontrar en una CPU, es de especial utilidad en operaciones masivamente paralelas, pudiendo llegar a proporcionar ganancias muy superiores con respecto al uso de la CPU.\\

Por otro lado, los algoritmos y técnicas de \textit{Soft Computing} se corresponde con una rama de la Inteligencia Artificial en la que no podemos calcular soluciones exactas en tiempo polinómico y/o en los que la información es incompleta, incierta o inexacta.\\

El propósito de este trabajo de fin de grado es la implementación en CUDA de algunos de estos modelos de \textit{Soft Computing} y, tras evaluar varias opciones, se optó por desarrollar dos: los mapas autoorganizados de \textit{Kohonen} \cite{kohonensom} y los árboles de decisión \cite{arbol}. Además, para evaluar el rendimiento de las implementaciones desarrolladas, usaremos conjunto de datos con un elevado número de muestras, combinando el uso de \textit{CUDA} y el \textit{framework} para clústers de computación distribuida \textit{Spark} \cite{spark}.\\

En definitiva, en este documento, explicaremos dichos modelos, analizaremos sus posibilidades de paralelización, realizaremos las implementaciones asociadas y evaluaremos los resultados obtenidos.

\section{Estado del arte: trabajos relacionados.}
Tanto la paralelización de los mapas autoorganizados de Kohonen como la de los árboles de decisión son problemas que han sido previamente estudiados para su paralelización en CUDA.\\

En \textbf{\textit{Parallel High Dimensional Self Organizing Maps Using CUDA}}  Codevilla, Bothelo, Filho y Gaya \cite{cudasomonline} proponen una implementación en CUDA para la formulación tradicional del mapa autoorganizado de Kohonen. En ella, proponen una versión en la que cada iteración se realiza en 3 fases. Una primera en la que con un valor \textit{p} arbitrario menor que el número de hebras por bloque que indica cuantos ``pasos'' debe realizar una hebra para el cálculo de la distancia euclídea, una reducción para encontrar la mejor distancia y una adaptación de pesos de neuronas basada en la dimensión del problema. \\

En \textbf{\textit{Parallel Batch Self-Organizing Map on Graphics Processing Unit Using CUDA}}  Daneshpajouh, Delisle, Boisson, Krajecki y Zakaria \cite{cudasombatch} plantean una adaptación en CUDA para la versión iterativa de cómputo en \textit{batchs} del mapa autoorganizado de Kohonen. En ella aprovechan las capacidades de concurrencia disponibles en los dispositivos CUDA, paralelizando parte del algoritmo y dejando la fase de adaptación de los pesos de neuronas para ser realizada en la CPU.

Con respecto a los árboles de decisión, \textbf{\textit{CUDT: a CUDA based decision tree algorithm}} , de Lo, Chang, Sheu, Chiu y Yuan \cite{cudt}, será la base de la implementación que nosotros vamos a realizar y se basa en el uso de la operación de la suma prefija, suma acumulada o \textit{scan} para resolver un cierto tipo de árboles de decisión específicos, en concreto, árboles de decisión cuyo objetivo es la clasificación de problemas con respuesta binaria.\\

Aparte de la aproximación por especialización presentada en el trabajo anterior, es otra alternativa, más frecuente y versátil en la variedad de problemas que puede resolver, la discretización de las variables utilizadas durante la construcción del árbol y el uso de histogramas para ello. Esto lo podemos ver en \textbf{\textit{Implementing Streaming Parallel Decision Trees on Graphic Processing Units}}, de Svantesson \cite{svatensson}, donde el objeto principal de su trabajo es paralelizar en CUDA los cálculos asociados a los histogramas utilizados en SPDT (Streaming Parallel Decision Trees) para generar un árbol.

\section{Objetivos.}
\begin{itemize}
    \item Iniciarse, estudiar y profundizar en el desarrollo de algoritmos paralelos en \textit{CUDA}.
    \item Analizar algoritmos de \textit{Soft Computing}, evaluando las capacidades que tienen para ser paralelizados.
    \item Implementar los algoritmos seleccionados en \textit{CUDA}.
    \item Combinar el uso de \textit{CUDA} y \textit{Spark} para resolver la paralelización masiva de problemas de forma eficiente.
    \item Utilizar conjuntos de datos de \textit{Big Data} que sean computacionalmente exigentes para el desarrollo de las pruebas.
    \item Realizar una evaluación de la calidad de los resultados obtenidos.
\end{itemize}

\section{Estructura del documento.}

\begin{itemize}
    \item En el primer capítulo, \textbf{Introducción y motivación}, hemos comentado los própositos para la realización de este trabajo y el grado de consecución de los objetivos planteados.
    \item En el segundo capítulo, \textbf{Modelos de Soft Computing considerados}, explicamos los fundamentos teóricos de los algoritmos de \textit{Soft Computing} que hemos decidido paralelizar.
    \item En el tercer capítulo, \textbf{Implementación}, comentamos el proceso de desarrollo seguido así como explicamos las soluciones finales implementadas y comentamos algunas de las alternativas y problemas que surgieron durante la realización de las implementaciones.
    \item En el cuarto capítulo, \textbf{Desarrollo de pruebas y análisis de resultados}, indicamos qué pruebas se han realizado, mostramos los resultados obtenidos y analizamos en profundidad las implicaciones de los mismos.
    \item En el último capítulo, \textbf{Conclusiones y trabajos futuros}, finalizamos el trabajo destacando las implicaciones más importantes de los resultados obtenidos y mostramos posibles alternativas para amplíar nuestro trabajo.
\end{itemize}

%\chapter{Tecnologías utilizadas.}
\section{CUDA.}
CUDA \textit{(Computer Unified Device Arquitecture)} \cite{cuda} es una tecnología propietaria desarrollada por \textit{NVIDIA} y lanzada en junio de 2007. Esta, proporciona al usuario un paradigma de programación general basado en C y que puede ser utilizado desde C/C++ y Fortran y, mediante el uso de algunas librerías/\textit{wrappers}, en otros muchos lenguajes, como Java o Python y, cuyo objetivo, es facilitar el desarrollo de código masivamente paralelo utilizando las GPUs de la misma compañía.\\

El modelo se fundamenta, en el desarollo de pequeñas ``funciones'', denominadas \textit{kernels}, con cierta similitud a una función normal de C, en la que se implementa el código que debe de realizar cada núcleo de la tarjeta gráfica. Al invocarse dicho \textit{kernel}, se indica el número de hebras que han de ejecutar dicho código. 

\subsection{Hebras, bloques y grids.}
Las \textbf{hebras} son la mínima unidad en la arquitectura CUDA.  Todas las hebras que se estén ejecutando en el mismo \textit{Streaming MultiProcessor (SM)} estarán ejecutando el mismo código. Cada uno de los SM del dispositivo tiene asociados una cantidad determinada de registros, su propia memoria caché, núcles y planificador, entre otras cosas.

CUDA hace que un mínimo de 32 hebras, denominado \textit{warp}, ejecuten instrucciones a la vez, aunque se hagan cálculos innecesarios.\\

Un \textbf{bloque} es un conjunto de hebras que van a ejecutar el mismo \textit{kernel}. Cuando invocamos un \textit{kernel}, hemos de lanzar como mínimo un bloque con N hebras. Todas las hebras de un bloque son ejecutadas por el mismo \textit{Streaming MultiProcessor}.\\

Por último, el \textbf{grid} o malla, es la abstracción máxima en CUDA y representa el conjunto de los bloques que ejecutan el \textit{kernel}. \\

Tanto las hebras como los bloques tiene un sistema de indexación que nos permiten saber en tiempo de ejecución el índice de la hebra y del bloque en el que se encuentra la hebra, permitiéndonos así repartir el trabajo de la manera que deseemos.

\subsection{La memoria compartida.}
Dentro de la tarjeta gráfica, nos encontramos con distintos niveles de memoria. Una vez los datos necesarios han sido traspasados del \textit{host (CPU)} al dispositivo a través del bus PCI-e x16 3.0 (en mi caso), esos datos son almacenados en una memoria DRAM de propósito general del dispositivo. Cuando un \textit{kernel} solicita datos de esta memoria, de manera similar a como ocurre en una CPU, los datos solicitados y los colidantes en memoria son colocados a través de varios niveles de caché, que tiene tamaño más limitado que la memoria DRAM pero el acceso y la escritura de los mismos son mucho más rápidos.\\

Una de las optimizaciones más habituales y que aumento de rendimiento conlleva (cuando es posible utilizarla) es el uso de la \textbf{memoria compartida}. Dicha memoria, es una región especial de la caché asociada a un bloque. Si se da una situación en la que podemos aprovechar que necesitemos acceder a los mismos datos dentro de las hebras de un bloque, es fundamental utilizar dicha memoria para obtener buenos resultados en cuanto a velocidad de ejecución se refiere. En el cuadro \ref{tab:cudamemory}, podemos ver un cuadro resumen de los tipos de memoria existentes, dónde se pueden usar y dónde se encuentran dichos datos en el dispositivo.

\begin{table}[h]
\begin{tabular}{|c|c|c|c|}
\hline
\textbf{Memoria}    & \textbf{Localización}                                           & \textbf{\begin{tabular}[c]{@{}c@{}}Acceso\\ (E = Escribir)\\ (L = Leer)\end{tabular}} & \textbf{\begin{tabular}[c]{@{}c@{}}Existente\\ hasta fin\\ de\end{tabular}} \\ \hline
\textbf{Registro}   & Caché                                                           & Kernel (E/L)                                                                          & Hebra                                                                       \\ \hline
\textbf{Local}      & \begin{tabular}[c]{@{}c@{}}DRAM\\ (Caché tras uso)\end{tabular} & Kernel (E/L)                                                                          & Hebra                                                                       \\ \hline
\textbf{Compartida} & Caché                                                           & Kernel (E/L)                                                                          & Bloque                                                                      \\ \hline
\textbf{Global}     & \begin{tabular}[c]{@{}c@{}}DRAM\\ (Caché tras uso)\end{tabular} & \begin{tabular}[c]{@{}c@{}}Host (E/L)\\ Kernel (E/L)\end{tabular}                     & \begin{tabular}[c]{@{}c@{}}Aplicación\\ o uso de free\end{tabular}          \\ \hline
\textbf{Constante}  & \begin{tabular}[c]{@{}c@{}}DRAM\\ (Caché tras uso)\end{tabular} & \begin{tabular}[c]{@{}c@{}}Host (E/L)\\ Kernel (L)\end{tabular}                       & \begin{tabular}[c]{@{}c@{}}Aplicación \\ o uso de free\end{tabular}         \\ \hline
\end{tabular}
\caption{Tabla resumen de los tipos de memoria en CUDA.}
\label{tab:cudamemory}
\end{table}

\subsection{Python: Numba y CuPy.}
Para desarrollar el código asociado a este proyecto, hemos optado por utilizar \textbf{Python} en vez de los tradicionales C o C++. \\

\textbf{Numba} \cite{numba} es un paquete para Python cuyo objetivo es la aceleración compilado fragmentos de código utilizando el compilador LLVM y dando la oportunidad de paralelizar código tanto para la CPU como para la GPU. En concreto, para las GPUs CUDA, proporciona al usuario un subconjunto de las características de CUDA con un nivel de abstracción mayor. Con eso no sólo conseguimos poder trabajar con CUDA desde Python sino, también evitar, si lo deseamos, manejar los traspasos de memoria entre host y dispositivo o la necesidad de indicar todos los tipos a la hora de inicializar un \textit{kernel} entre otras ventajas.
\begin{code}
\begin{minted}{python}
from numba import cuda
import numpy as np
# Definimos el kernel
@cuda.jit
def aumentar_en_1(un_array):
	# Cogemos el índice de la hebra
    pos = cuda.grid(1)

    # Si el índice está en el rango del array
    # incrementamos su valor
    if pos < un_array.size:
        un_array[pos] += 1

if __name__ == '__main__':
	# Declaramos un array de 10000 ceros
	ejemplo = np.zeros(10000)
	# Calculamos el número de bloques necesario
	bloques = ejemplo.size // 128 + 1
	# Lanzamos el kernel con bloques de 128 hebras
	aumentar_en_1[bloques, 128](ejemplo)
\end{minted}
\captionof{listing}{Kernel para incrementar en 1 los elementos de un array.\\\\}
\label{code:numbaexample}
\end{code}

\textbf{CuPy} \cite{cupy} es otro paquete de Python que, por un lado y de manera similar a Numba, nos permite generar kernels para CUDA en este caso de manera similar a los de C/C++ así como facilidades para generar kernels en los que se implementa reducciones u operaciones elemento a elemento en un array. Por otro lado, proporciona una API similar a la de NumPy pero las operaciones están implementadas utilizando CUDA. Además, CuPy está implementado de manera que permite utilizar directamente sus estructuras de datos sobre kernels de Numba, lo que nos permite combinar elementos de ambos paquetes según nos interese.

\subsection{Algunas operaciones relevantes.}

%%%%%%%%%% v2 %%%%%%%%%%%%%%%
CUDA hace que un mínimo de 32 hebras, denominado \textit{warp}, ejecuten instrucciones a la vez, aunque se hagan cálculos innecesarios.\\

Un \textbf{bloque} es un conjunto de hebras que van a ejecutar el mismo \textit{kernel}. Cuando invocamos un \textit{kernel}, hemos de lanzar como mínimo un bloque con N hebras. Todas las hebras de un bloque son ejecutadas por el mismo \textit{Streaming MultiProcessor}.\\

Por último, el \textbf{grid} o malla, es la abstracción máxima en CUDA y representa el conjunto de los bloques que ejecutan el \textit{kernel}. \\

Tanto las hebras como los bloques tiene un sistema de indexación que nos permiten saber en tiempo de ejecución el índice de la hebra y del bloque en el que se encuentra la hebra, permitiéndonos así repartir el trabajo de la manera que deseemos.

\subsection{La memoria compartida.}
Dentro de la tarjeta gráfica, nos encontramos con distintos niveles de memoria. Una vez los datos necesarios han sido traspasados del \textit{host (CPU)} al dispositivo a través del bus PCI-e x16 3.0 (en mi caso), esos datos son almacenados en una memoria DRAM de propósito general del dispositivo. Cuando un \textit{kernel} solicita datos de esta memoria, de manera similar a como ocurre en una CPU, los datos solicitados y los colidantes en memoria son colocados a través de varios niveles de caché, que tiene tamaño más limitado que la memoria DRAM pero el acceso y la escritura de los mismos son mucho más rápidos.\\

Una de las optimizaciones más habituales y que aumento de rendimiento conlleva (cuando es posible utilizarla) es el uso de la \textbf{memoria compartida}. Dicha memoria, es una región especial de la caché asociada a un bloque. Si se da una situación en la que podemos aprovechar que necesitemos acceder a los mismos datos dentro de las hebras de un bloque, es fundamental utilizar dicha memoria para obtener buenos resultados en cuanto a velocidad de ejecución se refiere. En el cuadro \ref{tab:cudamemory}, podemos ver un cuadro resumen de los tipos de memoria existentes, dónde se pueden usar y dónde se encuentran dichos datos en el dispositivo.

\begin{table}[h]
\begin{tabular}{|c|c|c|c|}
\hline
\textbf{Memoria}    & \textbf{Localización}                                           & \textbf{\begin{tabular}[c]{@{}c@{}}Acceso\\ (E = Escribir)\\ (L = Leer)\end{tabular}} & \textbf{\begin{tabular}[c]{@{}c@{}}Existente\\ hasta fin\\ de\end{tabular}} \\ \hline
\textbf{Registro}   & Caché                                                           & Kernel (E/L)                                                                          & Hebra                                                                       \\ \hline
\textbf{Local}      & \begin{tabular}[c]{@{}c@{}}DRAM\\ (Caché tras uso)\end{tabular} & Kernel (E/L)                                                                          & Hebra                                                                       \\ \hline
\textbf{Compartida} & Caché                                                           & Kernel (E/L)                                                                          & Bloque                                                                      \\ \hline
\textbf{Global}     & \begin{tabular}[c]{@{}c@{}}DRAM\\ (Caché tras uso)\end{tabular} & \begin{tabular}[c]{@{}c@{}}Host (E/L)\\ Kernel (E/L)\end{tabular}                     & \begin{tabular}[c]{@{}c@{}}Aplicación\\ o uso de free\end{tabular}          \\ \hline
\textbf{Constante}  & \begin{tabular}[c]{@{}c@{}}DRAM\\ (Caché tras uso)\end{tabular} & \begin{tabular}[c]{@{}c@{}}Host (E/L)\\ Kernel (L)\end{tabular}                       & \begin{tabular}[c]{@{}c@{}}Aplicación \\ o uso de free\end{tabular}         \\ \hline
\end{tabular}
\caption{Tabla resumen de los tipos de memoria en CUDA.}
\label{tab:cudamemory}
\end{table}

\subsection{Python: Numba y CuPy.}
Para desarrollar el código asociado a este proyecto, hemos optado por utilizar \textbf{Python} en vez de los tradicionales C o C++. \\

\textbf{Numba} \cite{numba} es un paquete para Python cuyo objetivo es la aceleración compilado fragmentos de código utilizando el compilador LLVM y dando la oportunidad de paralelizar código tanto para la CPU como para la GPU. En concreto, para las GPUs CUDA, proporciona al usuario un subconjunto de las características de CUDA con un nivel de abstracción mayor. Con eso no sólo conseguimos poder trabajar con CUDA desde Python sino, también evitar, si lo deseamos, manejar los traspasos de memoria entre host y dispositivo o la necesidad de indicar todos los tipos a la hora de inicializar un \textit{kernel} entre otras ventajas.
\begin{code}
\begin{minted}[fontsize=\footnotesize]{python}
from numba import cuda
import numpy as np
# Definimos el kernel
@cuda.jit
def aumentar_en_1(un_array):
  # Cogemos el índice de la hebra
    pos = cuda.grid(1)

    # Si el índice está en el rango del array
    # incrementamos su valor
    if pos < un_array.size:
        un_array[pos] += 1

if __name__ == '__main__':
  # Declaramos un array de 10000 ceros
  ejemplo = np.zeros(10000)
  # Calculamos el número de bloques necesario
  bloques = ejemplo.size // 128 + 1
  # Lanzamos el kernel con bloques de 128 hebras
  aumentar_en_1[bloques, 128](ejemplo)
\end{minted}
\captionof{listing}{Kernel para incrementar en 1 los elementos de un array.\\\\}
\label{code:numbaexample}
\end{code}

\textbf{CuPy} \cite{cupy} es otro paquete de Python que, por un lado y de manera similar a Numba, nos permite generar kernels para CUDA en este caso de manera similar a los de C/C++ así como facilidades para generar kernels en los que se implementa reducciones u operaciones elemento a elemento en un array. Por otro lado, proporciona una API similar a la de NumPy pero las operaciones están implementadas utilizando CUDA. Además, CuPy está implementado de manera que permite utilizar directamente sus estructuras de datos sobre kernels de Numba, lo que nos permite combinar elementos de ambos paquetes según nos interese.

%%%%%%%%%%%%%%%%%%%%%%%%%%%% AUX %%%%%%%%%%%%%%%%%%%%%%%%%%%%%%%%%%%%%%%%%%%%%%%%%%%%%%%%%%%%%

\subsection{Desarrollo del mapa autoorganizado online.}
En la versión \textit{online} del algoritmo, durante una serie de iteraciones se evalúa una muestra, se calcula su distancia con respecto a la matriz de pesos de las neuronas y se actualiza la región alrededor de la neurona más parecida. Un primer análisis, sobre este algoritmo nos indica una clara limitación a la hora de paralelizarlo pues existe una secuencialidad para evaluar una muestra. Tras este primer análisis, también dividir el algoritmo en los pequeños subproblemas que debemos de resolver:

\begin{itemize}
  \item 1. Inicialización aleatoria de la matriz de pesos.
  \item 2. Cálculo de las distancias euclídeas entre una muestra y los pesos.
  \item 3. Encontrar la BMU, es decir, la neurona cuyo vector de pesos era más próximo a la solución.
  \item 4. Actualizar la BMU y su vecindario.
  \item 5. Actualizar los parámetros de control.
\end{itemize} 

\subsubsection{Inicialización aleatoria de la matriz de pesos.}

Para generar la matriz de pesos, de manera eficiente, sobre todo si las dimensiones son relativamente grandes, hemos optado por hacerlo en la GPU. De esta manera, con tan sólo reservar el espacio de memoria en el dispositivo y realizar dicha inicialización en el dispositivo, nos ahorramos el tiempo que conllevaría generarla en la CPU y la transferencia de esos datos desde el host hasta el dispositivo. El uso de los paquetes Numba y CuPy, nos permite utilizar generadores de aleatorios de forma sencilla y cómoda. Numba proporciona una serie de funciones que podemos utilizar dentro de sus kerneles mientras que CuPy nos proporciona una interfaz similar a la del módulo \textit{random} de NumPy pero utilizando un \textit{wrapper} a la librería cuRAND para generar aleatorios eficientemente en CUDA. Además, ambas nos permiten establecer una semilla permitiéndonos así poder reproducir los experimentos que desarrollemos. Por la sencillez a la hora de utilizarlo, hemos decidido utilizar CuPy.

\begin{code}
\begin{minted}[fontsize=\footnotesize]{python}
import cupy as cp
weights = cp.random.ranf((rows, cols, d), dtype=cp.float32)
\end{minted}
\captionof{listing}{CuPy: Inicializar aleatoriamente un ndarray en la GPU.\\}
\label{code:cupyrandom}
\end{code}


Como podemos observar en el código fuente \ref{code:cupyrandom} con tan sólo indicar en una tupla las dimensiones deseadas y el tipo de dato que queremos (en este caso, reales en coma flotante de 32 bits cada uno) podemos solucionar este subproblema de manera sencilla y eficiente.\\

\subsubsection{Cálculo de las distancias euclídeas entre una muestra y los pesos.}
Nuestro objetivo ahora, es calcular la distancia euclídea entre una muestra y los pesos de todas las neuronas posibles que haya. Para resolverlo, hemos considerado que cada hebra que se lance se encargue de calcular la distancia entre una neurona y la muestra. Puesto que todas las hebras lanzadas van a utilizar la misma muestra hemos optado por que la primera hebra de cada bloque introduzca la muestra con la que se va a trabajar en memoria compartida para que el acceso sea más rápido. Otra posible opción, habría sido haber introducido o sólo los pesos o los pesos con la muestra, siempre y cuando todos los datos quepan en memoria compartida que, en la mayoría de dispositivos CUDA es de un máximo de 48 KB por \textit{Streaming MultiProcessor}. La raíz cuadrada de la distancia euclídea no es calculada puesto que afecta a la relación de orden, que es nuestro objeto de interés.

$$
sii \; a < b \leftrightarrow \sqrt{a} < \sqrt{b}
$$

\begin{code}
\begin{minted}[fontsize=\footnotesize]{python}
@cuda.jit
def euclidean_distance(ids, samples, weights, out, d):
  idx = cuda.grid(1)
    # Fase 1: Ponemos el vector del array en memoria compartida
    shared_vector = cuda.shared.array(shape=0, dtype=numba.float32)
    if cuda.threadIdx.x == 0:
        for i in range(d):
            shared_vector[i] = samples[ids * d + i]
    cuda.syncthreads()

    # Fase 2: Calculamos la distancia euclídea
    if idx * d < weights.size:
        distance = 0
        for i in range(d):
            i_distance = shared_vector[i] - weights[idx*d+i]
            distance += i_distance * i_distance
            
        # Fase 3: Lo escribimos en el array de salida.
        out[idx] = distance
\end{minted}
\captionof{listing}{Numba: Distancia euclídea entre muestra y nueronas.\\}
\label{code:euclideaonline}
\end{code}


\subsubsection{Encontrar la BMU, es decir, la neurona cuyo vector de pesos era más próximo a la solución}

\begin{code}
\begin{minted}[fontsize=\footnotesize]{python}
import cupy as cp
weights = cp.random.ranf((rows, cols, d), dtype=cp.float32)
\end{minted}
\captionof{listing}{CuPy: Reducción para encontrar el mínimo.\\}
\label{code:cupyreduction}
\end{code}

Para resolver este subproblema se ha utilizado la técnica de la \textbf{reducción.} La reducción es un algoritmo altamente utilizado tanto en CUDA como en otros entornos de programación paralela y, presente, por tanto, en librerías como \textit{thrust} (en C++) o \textit{CuPy} en Python para su uso de forma simple y cómoda. La idea de este algoritmo es utilizar la arquitectura paralela del dispositivo para realizar una serie de operaciones binarias sobre una serie de valores y obtener un único valor, donde la operación binaria cumple la propiedad asociativa. El ejemplo más habitual de este tipo de casos de uso es realizar la sumatoria de los elementos en un array. Para realizar esto en CUDA, cada hebra se encarga de una de estas operaciones binarias, este proceso es de nuevo realizado sobre los resultados obtenidos en el primer pase y reiterado hasta obtener un único resultado. \\


\begin{figure}[ht]
\centering
\includegraphics[scale=0.5]{imagenes/parallel_reduce.png}
\caption{Una reducción paralela de una sumatoria en CUDA.}
\label{image:cudareduction}
\end{figure}


En la figura \ref{image:cudareduction} podemos observar un ejemplo en el que se ve la estrucutra de cómo sería una reducción paralela en CUDA para un pequeño ejemplo. Para más información sobre cómo realizar una implementación de alto rendimiento de esta operación en CUDA puede consultarse en \cite{reduction}.\\

En nuestro trabajo la operación a la que queremos aplicar la reducción es el mínimo sobre un largo array de distancias. La operación mínimo \textbf{cumple la propiedad asociativa}, es decir:

$$min(min(a,b), c) = min(a, min(b, c))
$$

Además, si en vez pasar el valor para realizar la operación utilizamos un puntero en la memoria del dispositivo. Tras realizar toda la reducción con utilizar aritmética de punteros y restar al puntero inicial el puntero del valor mínimo obtenemos la solución deseada de forma eficiente. Una vez comprendido lo que esta operación implica, hemos decidido utilizar la función argmin de la librería CuPy que actúa de la forma comentada pero se encuentra altamente optimizada ofreciéndonos un gran rendimiento.

\begin{code}
\begin{minted}[fontsize=\footnotesize]{python}
import cupy as cp
min_index = cp.argmin(distances)
\end{minted}
\captionof{listing}{CuPy: Reducción para encontrar índice mínimo.\\}
\label{code:cupyreduce}
\end{code}


\subsubsection{Actualizar la BMU y su vecindario.}
Para esta operación necesitamos una vez tenemos encontrada la BMU actualizar una región alrededor de la misma y dependiente del parámetro de control $\sigma$, que controlaba el tamaño del vecindario. Recordemos que el parámetro $\sigma$ empezaba con un valor muy alto que se iba reduciendo exponencialmente y luego era fijado a otro parámetro indicado por el usuario, que debería de ser muy pequeño. Esto hace que en las primeras iteraciones se actualicen muchas neuronas y conforme vamos avanzando bastante menos. Los beneficios a la hora de paralelizar la actualización del mapa de neuronas de ese valor de $\sigma$, pues a mayor valor mayor es el número de neuronas alrededor que debemos evaluar. En esta implementación, hemos considerado siempre utilizar \textit{CUDA} para realizar la actualización pero sería una alternativa perfectamente viable utilizar la CPU en los casos en los que $\sigma$ es bajo y sólo un número pequeño de neuronas es afectado y el traslado de los datos necesarios de la matriz de pesos, que se encuentra en la GPU, de dispositivo a host y luego de vuelta no suponga un \textit{overhead} demasiado caro.\\

En la implementación que nosotros hemos realizado, que se corresponde a otro kernel de \textit{Numba} (código fuente \ref{code:updateonline}), cada hebra lanzada se corresponde a una neurona. Cada neurona comprueba su distancia con respecto a la BMU y si dicha distancia es válida procede a actualizar los datos que le corresponden conforme a la ecuación de actualización de pesos.

\begin{code}
\begin{minted}[fontsize=\footnotesize]{python}
@cuda.jit
def bmu_update(ids, samples, weights, d, bmu_row, 
    bmu_col, cols, eta, sigma_squared):
  idx = cuda.grid(1)
      if idx * d < weights.size:
          
          # 1. Medimos la distancia en la matriz del elemento actual a la BMU
          
          current_row = idx // cols
          current_col = idx % cols

          d_f = (current_col - bmu_col) * (current_col - bmu_col)
          d_f += (current_row - bmu_row) * (current_row - bmu_row)
          if d_f <= sigma_squared:
              # 2. Actualizamos acorde a esa distancia y el valor de sigma
              d_f = math.exp(-d_f/(2*sigma_squared))
              for i in range(d):
                  weights[idx * d + i] += eta * d_f * \
                  (samples[ids * d + i] - weights[idx * d + i])
\end{minted}
\captionof{listing}{Numba: Actualización de pesos de neuronas}
\label{code:updateonline}
\end{code}

\subsubsection{Actualizar los parámetros de control.}
Los parámetros de control $\eta$ y $\sigma$ son sólo dos parámetros a modificar por iteración y cuyo cálculo no es excesivamente complejo, razón por la que no es lo ideal paralelizar esta subproblema en la GPU y esto será realizando en la CPU, que, además puede realizar dicha operación de manera asíncrona mientras se está ejecutando alguno de los \textit{kernels}.

\subsubsection{Diagrama de flujo de la solución implementada.}
\begin{figure}[ht]
\centering
\includegraphics[scale=0.35]{imagenes/flujosomonline.png}
\caption{Diagrama de flujo para el mapa autoorganizado online.}
\end{figure}


\subsection{Desarrollo del mapa autoorganizado batch.}
En la versión \textit{batch} del algoritmo, aparece la posibilidad de evaluar múltiples muestras a la vez con el cambio de la ecuación de actualización de pesos. Ahora, en cada iteración los pesos de una neurona son la media de las muestras que lo activan y las que activan a las neuronas cercadas ponderada según la distancia en el vecindario del mapa. De manera similar a la sección anterior podemos subdividir el problema en pequeños subproblemas a resolver pero paralelizando en función del número de muestra en vez del de neuronas.

\begin{itemize}
  \item 1. Inicialización aleatoria de la matriz de pesos.
  \item 2. Cálculo de las distancias euclídeas entre todas las muestras y los pesos.
  \item 3. Encontrar la BMU para cada muestra.
  \item 4. Actualizar la matriz de pesos.
  \item 5. Actualizar los parámetros de control.
\end{itemize}

Tanto la \textbf{inicialización de la matriz de pesos} como la \textbf{actualización de los parámetros de control} son exáctamente \textbf{idénticas a la versión anterior} con ligeras variaciones, como que, por ejemplo, en la versión \textit{batch} no utilizamos la tasa de aprendizaje $\eta$ porque no es necesaria para obtener buenos resultados.\\

Para el \textbf{cálculo de las distancias euclídeas} necesitamos cambiar la idea planteada anteriormente puesto que ahora tenemos que evaluar todas las muestras con todas las neuronas. 
En este caso se lanzan tantas hebras como sea el producto del número de neuronas y el número de muestras, calculando cada hebra la distancia euclídea entre la muestra y la neurona que les corresponden. De manera similar a la anterior versión, obviamos el cálculo de la ráiz cuadrada. Puesto que ahora tenemos que evaluar todas las muestras a la vez no hemos planteado uso alguno de la memoria compartida para permitir que la solución sea aplicable a problemas de diversas características, pero si sabemos que la dimensión de una muestra del conjunto de datos no es excesivamente grande podríamos utilizar la memoria compartida para trabajar con los pesos del mapa. El código fuente \ref{code:euclideanbatch} muestra la implementación de este kernel en Numba.

\begin{code}
\begin{minted}[fontsize=\footnotesize]{python}
@cuda.jit
def batch_euclidean_distance(samples, weights, distances):
  idx = cuda.grid(1)
    if idx < distances.size:
        nrows, ncols, d = weights.shape
        nneurons = nrows * ncols
        row = idx // nneurons
        col = idx % nneurons
        wrow = col // ncols
        wcol = col % ncols

        my_distance = 0
        for i in range(d):
            i_distance = samples[row,i] - weights[wrow, wcol, i]
            my_distance += i_distance * i_distance
            
        distances[row, col] = my_distance
\end{minted}
\captionof{listing}{Numba: Distancia euclídea entre todas las muestras y neuronas.\\}
\label{code:euclideanbatch}
\end{code}

El proceso para \textbf{encontrar las BMU} vuelve a ser bastante similar utilizando la reducción. Sin embargo, ahora debemos de lanzar $N$ reducciones consecutivas para obtener la BMU de cada muestra. Utilizando CuPy podemos indicar si queremos hacer la reducción a lo largo de uno de los ejes, como se ve en el código fuente \ref{code:cupyreduce2}.

\begin{code}
\begin{minted}[fontsize=\footnotesize]{python}
import cupy as cp
# Se reduce cada fila
min_index = cp.argmin(distances, axis=1)
\end{minted}
\captionof{listing}{CuPy: Reducción para encontrar índice mínimo a lo largo de un eje en una matriz bidimensional.\\}
\label{code:cupyreduce2}
\end{code}

La parte más compleja de resolver en la versión \textit{batch} del algoritmo es la \textbf{actualización de la matriz de pesos}. Recordemos que en cada iteración, para obtener el nuevo vector de pesos hemos de aplicar la ecuación.

$$
 W_{i, j} = \frac{\sum_{k=0}^{f} \delta_f(c, [i,j]) \cdot X(T_k) }{\sum_{k=0}^{f} \delta_f(c, [i,j])}
$$\\

Por tanto, las neuronas que han sido activadas han de ser actualizadas en funcion de sus pesos. La solución que hemos planteado se basa en que primero se calculen las sumatorias del numerador y el denominador y posteriormente se pase otro kernel para hacer la división. Dado que al paralelizarlo existe la posibilidad de que múltiples hebras nos encontraríamos ante una condición de carrera lo que nos deja tres alternativas para resolver este problema.

\underline{Alternativas disponibles.}
\begin{itemize}
  \item a) Utilizar las operaciones de suma atómica del dispositivo.
  \item b) Utilizar una estructura auxiliar más grande y realizar reducciones sobre ella.
  \item c) Resolver este problema en la CPU.
\end{itemize}

La {opción a} fua la primera en ser implementada. El proceso se divide en dos \textit{kernels} uno que se encarga de generar los numeradores y denominadores para la fórmula de actualización nueva de cada neurona y otro que se encarga de hacer la división entre numerador y denominador para actualizar los pesos.\\

En el primero de estos \textit{kernels} (código fuente \ref{code:numbanumdem}) utilizamos la operación de suma atómica, esta operación es capaz de leer y escribir sobre una posición de memoria de la GPU en una única operación sin interferencia de ninguna otra hebra, lo que nos permite evitar el problema inicial limitando el rendimiento pero necesario para obtener resultados correctos. En la implementación propuesta, cada hebra se encargará de tomar la BMU de una muestra y actualizar su vecindario. 

\begin{code}
\begin{minted}[fontsize=\footnotesize]{python}
@cuda.jit
def prepare_update(bmu_row, bmu_col, samples, num, den, 
    nrows, ncols, sigma_squared):
  idx = cuda.grid(1)
    if idx < bmu_row.size:
        my_row = bmu_row[idx]
        my_col = bmu_col[idx]
        
        init_row = max(0, my_row - int(sigma_squared))
        finish_row = min(nrows, my_row + int(sigma_squared) + 1)
        
        init_col = max(0, my_col - int(sigma_squared))
        finish_col = min(ncols, my_col + int(sigma_squared) + 1)
        
    
        for i in range(init_row, finish_row):
            for j in range(init_col, finish_col):
                dist = (j-my_col) * (j-my_col) + (i-my_row) * (i-my_row)
    
                if dist <= sigma_squared:
                    hck = math.exp(-(dist)/(2 * sigma_squared))
                    cuda.atomic.add(den, i*ncols+j, hck)
                    d = samples.shape[1]
                    for k in range(d):
                        cuda.atomic.add(num, i*ncols*d + j*d +k,
                        hck * samples[idx, k])
\end{minted}
\captionof{listing}{Numba: Cálculo de numerador y denominador de la ecuación mediante suma atómica.\\}
\label{code:numbanumdem}
\end{code}

En el segundo kernel (código fuente \ref{code:numbaweightsbatch}), lanzamos tantas hebras como neuronas haya y actualizamos los pesos de cada neurona en función de los numeradores y denominadores calculados en el kernel anterior. Si esa neurona no se ha visto afectada en el cálculo de numeradores y denominadores, se mantiene los pesos de la iteración anterior.
\begin{code}
\begin{minted}[fontsize=\footnotesize]{python}
@cuda.jit
def finish_update(weights, num, den):
  idx = cuda.grid(1)
    if idx < den.size:
        nrows, ncols, d = weights.shape
        row = idx // ncols
        col = idx % ncols
        my_den = den[row * ncols + col]
        if my_den != 0:
            for k in range(d):
                weights[row, col, k] = num[row*ncols*d + col*d +k] / my_den
      
        cuda.syncthreads()
       
        for k in range(d):
            num[row*ncols*d + col*d +k] = 0
            
        den[row * ncols + col] = 0
\end{minted}
\captionof{listing}{Numba: Actualización de pesos en función de numeradores y denominadores.\\}
\label{code:numbaweightsbatch}
\end{code}


El procedimiento será mucho más costoso en las primeras iteraciones y menos costoso conforme se avance al irse reduciendo el vecindario. Para evaluar el rendimiento de esta versión utilizamos el profiler de NVIDIA, \textbf{nvprof} para medir el tiempo que se tarda en obtener los resultados de esta forma así como el tiempo que ha tardado en ejecutarse el algoritmo. Para realizar el profile hemos utilizado el conjunto de datos de las caras de \textit{Olivetti} (explicado en el siguiente capítulo). En la tabla \ref{tab:batchprofileparams} podemos observar los parámetros de control utilizados para la ejecución de este ejemplo.

\begin{table}[ht]
\begin{tabular}{|l|l|l|l|l|l|l|l|}
\hline
\multicolumn{1}{|c|}{\textbf{Muestras}} & \textbf{Dimensión} & \multicolumn{1}{c|}{\textbf{Neuronas}} & \textbf{Épocas} & \textbf{\begin{tabular}[c]{@{}l@{}}Épocas\\ 1º Fase\end{tabular}} & $\sigma_0$ & $\sigma_f$ & $\tau$ \\ \hline
400                                     & 4096               & 400 (20x20)                            & 400             & 100                                                               & 15         & 0.1        & 400    \\ \hline
\end{tabular}
\caption{Parámetros de control para el ejemplo utilizando en el profiler.}
\label{tab:batchprofileparams}
\end{table}


Los resultados más relevantes proporcionados por el profiler se encuentran en la tabla \ref{tab:batchprofile}.\\

\begin{table}[ht]
\centering
\begin{tabular}{|l|l|crrr}
\hline
\rowcolor[HTML]{EFEFEF} 
\multicolumn{1}{|c|}{\cellcolor[HTML]{EFEFEF}\textbf{Actividad}}                                                  & \multicolumn{1}{c|}{\cellcolor[HTML]{EFEFEF}\textbf{\begin{tabular}[c]{@{}c@{}}Tiempo\\ total\end{tabular}}} & \multicolumn{1}{c|}{\cellcolor[HTML]{EFEFEF}\textbf{\begin{tabular}[c]{@{}c@{}}Nº\\usos\end{tabular}}} & \multicolumn{1}{c|}{\cellcolor[HTML]{EFEFEF}\textbf{\begin{tabular}[c]{@{}c@{}}Mín\end{tabular}}} & \multicolumn{1}{c|}{\cellcolor[HTML]{EFEFEF}\textbf{\begin{tabular}[c]{@{}c@{}}Medio\end{tabular}}} & \multicolumn{1}{c|}{\cellcolor[HTML]{EFEFEF}\textbf{\begin{tabular}[c]{@{}c@{}}Máx\end{tabular}}} \\ \hline
\rowcolor[HTML]{EFEFEF} 
\cellcolor[HTML]{F8A102}\textbf{CPU + GPU}                                                                        & \cellcolor[HTML]{F8A102}56,62 s                                                                              & \multicolumn{1}{l}{\cellcolor[HTML]{EFEFEF}}                                                               & \multicolumn{1}{l}{\cellcolor[HTML]{EFEFEF}}                                                                  & \multicolumn{1}{l}{\cellcolor[HTML]{EFEFEF}}                                                                  & \multicolumn{1}{l}{\cellcolor[HTML]{EFEFEF}}                                                                  \\ \cline{1-2}
\rowcolor[HTML]{EFEFEF} 
\cellcolor[HTML]{38FFF8}\textbf{GPU}                                                                              & \cellcolor[HTML]{38FFF8}55,37 {[}100 \%{]}                                                                   & \multicolumn{1}{l}{\cellcolor[HTML]{EFEFEF}}                                                               & \multicolumn{1}{l}{\cellcolor[HTML]{EFEFEF}}                                                                  & \multicolumn{1}{l}{\cellcolor[HTML]{EFEFEF}}                                                                  & \multicolumn{1}{l}{\cellcolor[HTML]{EFEFEF}}                                                                  \\ \hline
\rowcolor[HTML]{ECF4FF} 
\textbf{\begin{tabular}[c]{@{}l@{}}Cálculo de\\ distancias\end{tabular}}                                          & 14,67 s {[}26,5 \%{]}                                                                                        & \multicolumn{1}{c|}{\cellcolor[HTML]{ECF4FF}400}                                                           & \multicolumn{1}{r|}{\cellcolor[HTML]{ECF4FF}35,67 ms}                                                         & \multicolumn{1}{r|}{\cellcolor[HTML]{ECF4FF}35,68 ms}                                                         & \multicolumn{1}{r|}{\cellcolor[HTML]{ECF4FF}45,58 ms}                                                         \\ \hline
\rowcolor[HTML]{ECF4FF} 
\textbf{\begin{tabular}[c]{@{}l@{}}Encontrar \\ BMUs\end{tabular}}                                                & 23,78 ms {[}0,04 \%{]}                                                                                       & \multicolumn{1}{c|}{\cellcolor[HTML]{ECF4FF}400}                                                           & \multicolumn{1}{r|}{\cellcolor[HTML]{ECF4FF}54,82 $\mu$s}                                                     & \multicolumn{1}{r|}{\cellcolor[HTML]{ECF4FF}59,47 $\mu$s}                                                     & \multicolumn{1}{r|}{\cellcolor[HTML]{ECF4FF}68,26 $\mu$s}                                                     \\ \hline
\rowcolor[HTML]{ECF4FF} 
{\color[HTML]{CB0000} \textbf{\begin{tabular}[c]{@{}l@{}}Generar \\ numeradores y \\ denominadores\end{tabular}}} & {\color[HTML]{FE0000} \textbf{39,97 s {[}72,19 \%{]}}}                                                       & \multicolumn{1}{c|}{\cellcolor[HTML]{ECF4FF}400}                                                           & \multicolumn{1}{r|}{\cellcolor[HTML]{ECF4FF}1,05 ms}                                                          & \multicolumn{1}{r|}{\cellcolor[HTML]{ECF4FF}99,93 ms}                                                         & \multicolumn{1}{r|}{\cellcolor[HTML]{ECF4FF}458,2 ms}                                                         \\ \hline
\rowcolor[HTML]{ECF4FF} 
\textbf{\begin{tabular}[c]{@{}l@{}}Actualizar \\matriz  de pesos\end{tabular}}                                    & 695,01 ms {[}1,26 \%{]}                                                                                      & \multicolumn{1}{c|}{\cellcolor[HTML]{ECF4FF}400}                                                           & \multicolumn{1}{r|}{\cellcolor[HTML]{ECF4FF}1,68 ms}                                                          & \multicolumn{1}{r|}{\cellcolor[HTML]{ECF4FF}1,74 ms}                                                          & \multicolumn{1}{r|}{\cellcolor[HTML]{ECF4FF}1,86 ms}                                                          \\ \hline
\end{tabular}
\caption{Resumen de tiempos más importantes de la ejecución del profiler.}
\label{tab:batchprofile}
\end{table}

Podemos observar que el cuello de botella en esta ejecución es el kernel asociado a computar los numeradores y denominadores, ocupando el 72 \% del tiempo de ejecución de la CPU. Por ello, vamos a analizar el resto de opciones para ver cuál es su rendimiento.

La \textbf{opción b} fue descartada pues conllevaría serían necesarias dos estructuras auxiliares para almacenar los numeradores y los denominadores. En el caso en cuestión, necesitaríamos 2,5 GiB para los numeradores y 625 KiB que generan una complejidad espacial excesiva.\\

Dados los resultados de la opción a, hemos decidido probar también a implementar la \textbf{opción c}. En esta versión, el cálculos de distancias y encontrar la BMU es realizado en la GPU y la actualización de pesos en la CPU con el fin de evaluar si el gran porcentaje de tiempo que se tarda es por la dificultad del problema o porque las operaciones atómicas del dispostivo \textit{CUDA} no son lo suficientemente eficientes. Tras evaluarlo empíricamente, el tiempo total de ejecución de esta opción alcanzó los 160,56 segundos que es mucho más lento que los 56,62 segundos de la opción a. Por tanto, esta opción también fue descartada y nos quedamos con la implementación inicial utilizando la suma atómica.


\subsubsection{Diagrama de flujo de la solución final implementada.}

\begin{figure}[ht]
\centering
\includegraphics[scale=0.35]{imagenes/flujosombatch.png}
\caption{Diagrama de flujo para el mapa autoorganizado batch.}
\end{figure}
%!TEX root = ../proyecto.tex
\chapter{Modelos de Soft Computing considerados.}
\section{Mapas auto-organizados \textit{(Self Organizing Map)}}
A principio de la década de los 80, el científico finlandés Teuvo Kohonen \cite{kohonensom}, planteó un modelo de aprendizaje automático no supervisado y competitivo basándose en el estudio del funcionamiento del córtex cerebral. El modelo planteado, denominado mapa auto-organizado, red auto-organizada o red neuronal de Kohonen, entre otros nombres similares, es una red neuronal artificial, y las principales características que la definen son las siguientes:\\

\begin{itemize}
	\item Es una \textbf{red neuronal artificial}. Esto quiere decir, a grandes rasgos, que la estructura que genera el modelo está basada en una red de múltiples unidades, llamadas ``neuronas'', que se encuentran interconectadas entre sí.
	
	\item La red neuronal de Kohonen tiene \textbf{dos capas}. Una capa de entrada, con tantas neuronas como características o atributos tenga el vector que representa a la muestra que vaya a ser evaluada por la red, es decir, tanto una muestra como la capa de entrada tendrán la misma dimensión y, una característica, podría ser, por ejemplo, la altura o el peso de una persona. La segunda capa es la capa competitiva o capa de Kohonen, de un tamaño a decidir por el usuario y una estructura habitualmente bidimensional, aunque podría perfectamente usarse cualquier otro número de dimensiones.

	\item Asociado a cada neurona de la capa competitiva, tenemos un vector de referencia variable $w_i(t) \in {\rm I\!R}^n, i=1,2, ..., k$, es un decir un vector de ``referencia'' con respecto a la representación vectorial de las muestras estadísticas $\vec{X} = \vec{X}(t) \in {\rm I\!R}^n$ que se corresponden al conjunto de datos de entrada para el entrenamiento del modelo. Estos vectores también son denominados vectores de pesos y, al conjunto de ellos, se le denomina \textit{codebook}. En los casos en los que el algoritmo sea aplicado para realizar clustering, dicho vector representará el valor promedio de los vectores de características de las muestras asociadas al clúster de su neurona. 

	\item Asociado a cada neurona de la capa competitiva, tenemos un vector de pesos sinápticos obtenido a través de las conexiones con la capa de entrada, que es modificado durante el proceso de aprendizaje. A dicho vector de pesos se le llama vector de referencia, y representa el valor promedio de la categoría asociada a esa neurona. El conjunto de todos esos vectores de referencia es denominado \textit{codebook}.

	\item Es un algoritmo \textbf{no supervisado}, es decir, es capaz de encontrar patrones comunes basándose en los datos de la muestra de entrada sin necesidad de que, cuando una muestra entre a la red, se indique a qué categoría pertenece.

	\item Es un modelo \textbf{competitivo}. Cuando se recibe una muestra, todas las neuronas compiten por ser activadas pero sólo la mejor será activada.
\end{itemize}

\begin{figure}
\centering
\includegraphics[width=0.8\textwidth]{imagenes/arquitectura_som.png}

\caption{Esquema de una red neuronal de Kohonen.}
%\textit{Autor: Damian Jankowski}
\end{figure}

\subsection{Proceso de entrenamiento.}
\textit{Nota: Por simplicidad en la notación y en la explicación vamos a considerar que la capa competitiva es unidimensional. La explicación propuesta es extensible para un número de dimensiones arbitrario realizando los cálculos relacionados con la posición del vector de referencia dentro del codebook y la posición de la BMU utilizando vectores con un valor para cada una de las dimensiones correspondientes.\\}

Sea $\vec{X}$ un vector, que representa una secuencia de muestras estadísticas de un observable $x=x(t) \in {\rm I\!R}$, donde $t$ es el instante o iteración de la extracción de la muestra y un conjunto de vectores de referencia variables (\textit{codebook}), $w_i(t)$, que representaremos de forma abreviada como el vector $\vec{W(t)}: w_i(t) \in {\rm I\!R} \; | i: 0, 1, 2, ..., n $, donde los $w_i(0)$ están inicializados a un valor aleatorio siguiendo una distribución uniforme entre 0 y 1, se realizan los siguientes pasos, de forma iterativa, hasta alcanzar el número de iteraciones máximo $\lambda$, que determina el usuario como criterio de terminación para el proceso de entrenamiento.\\

En primer lugar, una muestra del vector $\vec{X}$ \textbf{será comparada simultáneamente con cada vector de referencia} $\vec{W(t)} = \{w_0(t), w_1(t), ..., w_i(t), ...,$ ${w_{n-1}}(t)\}$ en cada instante de tiempo $t$, utilizando para ello la distancia euclídea entre vectores, $||{X(t)w_i(t)}||_n$. Otras formulaciones para la distancia, como por ejemplo, la distancia de Mahalanobis, podrían haber sido usadas en su lugar. \\

A continuación, se determina \textbf{el vector de referencia más parecido a la muestra evaluada} en el instante $t$, y, a la ubicación que ocupa dicho vector de referencia (en índices) en el $codebook$, se le denomina \textbf{BMU \textit{(Best Matching Unit)}}.

$$ BMU(X(t)) = argmin_{w \in \vec{W}}{||Xw||}_n$$

La función $argmin$ devuelve la posición del vector en la que se alcanza el valor mínimo y, a la distancia entre la muestra del instante $t$ y su mejor vector de referencia la denominaremos

$$ D(X(t)) = min_{w \in \vec{W}}{||Xw||}_n$$. 

Una vez se ha seleccionado la BMU, se inicia un proceso de \textbf{actualización de los vectores de referencia} presentes en el vector $\vec{W}$. La obtención de un mejor ajuste para la siguiente iteración se realiza mediante la actualización del conjunto de vectores de referencia en función de la distancia entre la muestra y la BMU y la posición de la BMU en el vector $\vec{W}$. Este ajuste lo vamos a reflejar añadiendo a cada vector de referencia el término $\Delta W(t)$.

$$\vec{W}(t+1) = \vec{W}(t) + \Delta \vec{W}(t)$$

Durante este fase de actualización, tres parámetros de control pueden ser ajustados por el usuario. El parámetro de control $\tau$ tiene como objetivo marcar el ritmo al que los otros dos, el tamaño del vecindario y la tasa de aprendizaje, decrecen conforme avanza el tiempo siguiendo una función gaussiana. Es habitual utilizar, como $\tau$, el número de iteraciones del proceso de entrenamiento. Por las características de los otros parámetros de control, que vamos a explicar a continuación, un menor valor de $\tau$ permite que el algoritmo realice una transición más rápida de una fase de exploración brusca, en la que gran parte de los vectores de referencia son modificados, a una fase de refinamiento en los que sólo el vector de referencia de la BMU y, quizás, los vectores más próximos, son ligeramente actualizados.\\

El tamaño del vecindario, determina un radio alrededor de la posición de la BMU en el vector $\vec{W(t)}$, en el que los vectores de referencias van a sufrir algún cambio. Si los vectores de referencia no se encuentran dentro del subconjunto $\vec{W_A} = \{w_{BMU(j)-\sigma^2}, ..., w_{BMU(j)}, ..., w_{BMU(j)+ \sigma^2} \}$ su valor para $\delta_i(t)$ será 0. El parámetro $\sigma$ es inicializado en el instante $t=0$ por el usuario y actualizado conforme a una gaussiana. Además del sistema de decrecimiento exponencial presente en la gaussiana, posibilitamos que el usuario determine un valor fijo para $\sigma$, $\sigma_t$ que puede establecerse a partir de una iteración determinada por el mismo, a la que denominamos $z$.


$$\sigma(t) = \left\{
\begin{array}{ll}
\sigma_0e^{-\frac{t}{\tau}} & si \;\;t < z\\
\sigma_f & si  \;\; t\geq z
\end{array}
\right.
$$\\



El tamaño del vecindario, $\sigma$, es utilizado para el cálculo de una función de vecindario, que en función del parámetro $\sigma$ y la distancia (en índices) entre la BMU de cada vector de referencia $\vec{W(t)}$ pondera la modificación que va a ser realizada, permitiendo mantener propiedades topográficas y haciendo que, conforme nos alejamos (en índices) de la posición de la BMU en el vector $\vec{W}$ los cambios realizados sean menores. La función de vecindario, $\delta_f$, viene dada por:

$$\delta_i(t) = e ^{-\frac{||BMU(X(t))-(i)||}{2\sigma(t)^2}}. $$\\	


En la figura \ref{img:vecindario}, podemos observar una representación gráfica del vector $\vec{W}$ en un codebook bidimensional. Cada uno de los puntos rojos se corresponde con un vector de referencia, la BMU ha sido destacada de color amarillo y la región alrededor de la BMU que va a sufrir cambios significativos, el vecindario, queda delimitada con círculo alrededor de la BMU cuyo radio es $\sigma^2$.
\begin{figure}[H]
\centering
\includegraphics[width=0.3\textwidth]{imagenes/vecindario.png}

\caption{Representación gráfica del vecindario para un codebook bidimensional.}
\label{img:vecindario}
%\textit{Autor: Damian Jankowski}
\end{figure}


Por otro lado, la tasa de aprendizaje, $\eta$, tiene como objetivo ponderar la actualización de los pesos haciendo que, conforme avancen las iteraciones no sólo se vaya reduciendo el número de vectores de referencia que sufren cambios significativos, sino que, además, conforme avance el número de iteraciones incluso el vector de referencia de la BMU reciba cambios más ligeros. De manera similar al tamaño del vecindario, su valor inicial, $\eta_0$ es indicado por el usuario, y se puede establecer una iteración $z$ en la que es fijado a un valor determinado.

$$\eta(t) = \left\{
\begin{array}{ll}
\eta_0e^{-\frac{t}{\tau}} & si \;\;t < z\\
\eta_f & si  \;\; t\geq z
\end{array}
\right.$$\\

Combinando la función de vecindario, la tasa de aprendizaje y la distancia entre la BMU y la muestra a ser evaluada en el instante $t$, obtenemos la fórmula que actualiza los vectores de referencia.

$$\Delta{w_i(t)}=\eta(t)\delta_i(t)(X(t)-w_i(t))$$


Una vez alcanzado el número de iteraciones máximo, el vector $\vec{W}$ contiene los vectores de referencia de la solución de tal manera que si, el algoritmo estuviese siendo usado para resolver problemas de clustering, cada vector de referencia representaría un clúster.

\begin{figure}[H]
\centering
\includegraphics[width=1.0\textwidth]{imagenes/somtraining.png}
\caption{Proceso de entrenamiento de una red neuronal de Kohonen.}
%\textit{\space\space\space\space\space\space\space\space\space Licencia: CC BY-SA 3.0; Autor: Dan Stowell (MCLD)}
\label{img:somtraining}
\end{figure}

La figura \ref{img:somtraining} nos muestra, a grandes rasgos, un esquema del proceso de entrenamiento del mapa auto-organizado de Kohonen con una capa competitiva bidimensional.\\

En la primera parte de la figura, observamos la selección de la BMU. Para ello, del conjunto de datos de entrenamiento (sección azul) se selecciona una muestra de forma aleatoria (círculo blanco) y se toma como mejor neurona del mapa (cada neurona es una intersección de la malla de líneas y, la mejor, es resaltada con un círculo amarillo intenso) la neurona que está más cerca de la muestra.\\

En la segunda parte, observamos el proceso de actualización de los pesos de las neuronas. En este paso, la BMU más cercana se mueve hacia la posición de la muestra, y, las neuronas dentro de su vecindario (círculo amarillo menos intenso), se acercan también a la muestra pero en un menor grado.\\

Por último, vemos cómo, tras un número de iteraciones, el mapa es capaz de ofrecer una aproximación de la distribución de los datos.
\subsection{Usos del mapa auto-organizado.}
El modelo del mapa auto-organizado puede ser utilizado para diversas tareas, de entre las que destacan:\\

\begin{itemize}
	\item \textbf{Clustering} - es decir, generar agrupaciones del conjunto de datos de entrada. Por regla general, cada neurona de la capa de Kohonen representaría una posible agrupación de los datos. 

	\item \textbf{Visualización de datos de alta dimensionalidad.} Tras finalizar el proceso de entrenamiento, podemos utilizar diferentes técnicas para obtener una representación visual de las características topológicas de la muestras. Las matrices-U, las matrices-P o los planos de componentes son algunos de los modelos utilizados para visualizar el mapa auto-organizado.

	\item \textbf{Clasificación.} Una vez terminado el proceso de entrenamiento, pueden asignarse etiquetas a cada uno de los nodos y resolver problemas de clasificación dependiendo de qué BMU se active. 
	\end{itemize}

\subsection{Mapa auto-organizado batch.}
El proceso de entrenamiento previamente mencionado se corresponde al del mapa auto-organizado tradicional u \textit{online}. En ese proceso, durante una iteración, se evalúa un subconjunto de los datos como parte de un proceso secuencial de: encontrar la BMU y actualizar los pesos correspondientes. Posteriormente \cite{kohonenbatch}, basándose en las propiedades matemáticas del mapa auto-organizado \textit{online}, se derivó una formulación para realizar el proceso de actualización de pesos, en una sola iteración, para un bloque de muestras. Esta versión del algoritmo, es denominada \textbf{mapa auto-organizado \textit{batch}}. \\

En esta versión, la regla para la actualización de los vectores de referencia se procesan múltiples muestras del conjunto de datos de entrada, ya sea el conjunto de datos entero en cada iteración, como ocurre en nuestra implementación, o tomando un subconjunto aleatorio de muestras con reemplazo. Durante cada iteración, se recorren todas las muestras a ser evaluadas y, para cada una de ellas, se determina la BMU de forma similar a la versión \textit{online}. A continuación, procedemos a actualizar el conjunto de vectores de referencia de la siguiente manera:\\

1 - Para cada muestra, $X_j$ consideramos un subconjunto de los vectores de referencia del vector $\vec{W}$ alrededor de la BMU, al que denominamos $\vec{W_A} = \{w_{BMU(j)-\sigma^2}, ..., w_{BMU(j)}, ..., w_{BMU(j)+ \sigma^2} \}$\\

2 - Cada vector de referencia posee dos variables que nos permitirán obtener el vector de referencia para la próxima iteración mediante la realización del cociente entre ambas. Ambas serán inicializadas a 0 y cuando el vector de referencia en cuestión se encuentre dentro del rango $W_A$ para una muestra específica x, sumaremos al numerador el producto de la función de vecindario para la muestra X, su BMU y la posición que ocupa (en índices) el vector de referencia en el vector $\vec{w}$ y, en el denominaremos, sumaremos tan sólo el valor escalar de la función de vecindario. \\

3 - El cociente entre ambas conforma el valor del vector de referencia para la próxima iteración. Por tanto, hemos calculado nuestro nuevo vector de referencia como un promedio de las muestras de las que ha sido BMU y las muestras que han sido activadas en posiciones (en índices) del vector $\vec{W}$ cercanas ponderadas en función de esa distancia, dando más relevancia a las muestra más cercanas.\\

El uso del modelo \textit{batch} frente al modelo tradicional nos va a permitir paralelizar el algoritmo de forma mucho más eficiente, ya que, mientras que, en la versión \textit{online}, debíamos actualizar los vectores de referencia tras cada iteración, en la versión \textit{batch} podemos evaluar un conjunto de muestras de forma simultánea y, al estar evaluando múltiples muestras y por el nuevo método de actualización de los vectores de referencia converge en un número de iteraciones considerablemente inferior al necesario en el primero. Sin embargo, como comentan \textit{Jean-Claude Fort, Patrick Letremy y Marie Cottrell} \cite{compsom}, el método de entrenamiento utilizado en la versión \textit{batch} del algoritmo presenta una mayor dependencia de la inicialización de los vectores de pesos en el instante inicial, pudiendo proporcionar clusters muy desbalanceados o una peor representación de las características topográficas que la versión \textit{online}.

En el resto del documento nos referiremos a cada vector de referencia como pesos asociados a las neuronas y con estructura o ``matriz'' de pesos haremos referencia al vector $\vec{W}$


\subsection{Medidas de calidad.}
Para medir la calidad de un mapa auto-organizado una vez entrenado podemos utilizar dos medidas sobre el conjunto de muestras usado:\\

El \textbf{error medio de cuantificación} nos permite medir la precisión del mapa creado. Se calcula tomando la media de las distancias euclídeas entre cada una de las muestras y su correspondiente BMU.\\

$$
\epsilon_q = \frac{1}{|X|}\sum_{x \in X}{||x-w_{BMU(x)}||}
$$\\

El \textbf{error topográfico} mide la capacidad que ha tenido el modelo de conservar las propiedades topográficas del conjunto de muestras de entrenamiento. Podemos medir dicho error como:\\
$$
u(x = \left\{
\begin{array}{ll}
1 & si \; su \; BMU \; y \; la \; segunda \; BMU \; son \; adyacentes.\\
0 & en \; caso \; contrario.
\end{array}
\right.
$$
$$
\epsilon_t =  \frac{1}{|X|}\sum_{x \in \vec{X}} u(x)
$$\\

\section{Árboles de decisión.}
Un árbol de decisión \cite{arbol} es un modelo de aprendizaje automático supervisado utilizado para resolver problemas de clasificación y extensible para resolver problemas de regresión. Un árbol de decisión, una vez entrenado, está formado por una estructura jerárquica de reglas que nos indican a qué categoría pertenece una muestra del conjunto de datos de entrada. El árbol está formado por dos tipos de nodos:\\

\begin{itemize}
	\item Los \textbf{nodos de decisión}. En dichos nodos existe una pregunta sobre un atributo y valor (o varios) y, dependiendo de la respuesta, se procede a evaluar otro nodo del árbol.
	\item Los \textbf{nodos terminales} o nodos respuesta nos indican la clase o, el valor, en caso de árboles de regresión, a la que ha de pertenecer una muestra si, al ser evaluada, dicho nodo ha sido alcanzado. Estos nodos se corresponden con las hojas del árbol formado.
\end{itemize}


\subsection{Proceso de entrenamiento.} 

Obtener un árbol de decisión óptimo es un problema \textbf{NP-completo}, es decir, no se conoce una manera de resolver este problema con una complejidad de tiempo polinómico por lo que es habitual que los algoritmos que realizan el entrenamiento de este modelo sigan estrategias voraces \textit{(greedy)}. Los algoritmos más conocidos y utilizados para realizar esta tarea (ID3, CART, C4.5, C5.0) siguen un esquema de entrenamiento similar. \\

La idea que sigue el proceso de entrenamiento es realizar una serie de particiones binarias sobre el conjunto de datos inicial, calculando todos los posibles puntos de corte de la partición y evaluando cuál es el mejor de ellos. Este proceso es repetido hasta que se completa el árbol, es decir, todas las muestras han sido clasificadas, o, alguna de las condiciones de finalización temprana se cumpla, si es que la hubiere.\\

Un \textbf{punto de corte} es una combinación de un atributo del problema y un valor para el mismo con el que se va a particionar el conjunto de muestras siguiendo una relación de orden, quedando una partición para las muestras cuyo valor para el atributo sea inferior o igual al valor proporcionado y otra partición para las muestras con valores superiores (el caso de ser igual podría ser cambiado de la partición inferior a la superior siempre que se mantenga el criterio durante todo el proceso de entrenamiento y evaluación).\\

En las aproximaciones para entrenamiento de árboles de decisión en las que la evaluación de los puntos de corte se realiza de forma exhaustiva, es decir, analizando todos los posibles puntos de corte, y se trabaja sólo con atributos numéricos, los puntos de corte se calculan de la siguiente manera: para cada atributo y sus correspondientes valores proporcionados por las muestras, se ordenan los valores de forma ascendente o descendente y, si al realizar un recorrido secuencial sobre los valores el valor actual y su sucesor son diferentes, el punto medio entre ambos valores constituirá un punto de corte.

La calidad de cada uno de los puntos de corte obtenidos es evaluada conforme a diferentes criterios dependiendo del algoritmo, habitualmente basados en la impureza de cada una de las dos subdivisiones obtenidas al realizar el corte. \\

La \textbf{ganancia de información} es una de las posibles medidas para determinar el mejor punto de corte de todos los posibles y se basa en la siguiente fórmula:

$$
GI(D,s) = Impureza(D) - \frac{|D_{izq}|}{|D|} \cdot Impureza(D_{izq}) - \frac{|D_{der}|}{|D|} \cdot Impureza(D_{der})
$$

donde $D$ es el conjunto de datos de la partición actualmente considerada, $s$ es el punto de corte y $D_{izq}$ y $D_{der}$ son las subparticiones obtenidas a partir del punto de corte.\\

Una medida de \textbf{impureza} \cite{impurity} es una función que, dada un conjunto de datos, mide la cantidad de clases distintas que hay en ese conjunto. Dicha medida valdrá 0 si todos los elementos pertenecen a la misma clase y 1 si cada elemento es de una clase distintas. En la tabla \ref{tab:entropy} destacamos algunas de estas medidas.\\

\begin{table}[ht]
\centering
\begin{tabular}{|l|l|l|}
\hline
\textbf{Impureza} & \textbf{Tarea} & \textbf{Fórmula}                         \\ \hline
\textbf{Entropía}  & Clasificación  & $\sum_{i=1}^{C}- f_i \cdot log (f_i)$    \\ \hline
\textbf{Gini}      & Clasificación  & $\sum_{i=1}^{C} f_i (1 - f_i)$           \\ \hline
\textbf{Varianza}  & Regresión      & $\frac{1}{N}\sum_{i=1}^|D|(y_i - \mu)^2$ \\ \hline
\end{tabular}
\caption{Algunas medidas de impureza.}
\label{tab:entropy}
\end{table}

donde $f_i$ es la probabilidad de pertenecer a la clase $i$ en una división, $C$ es el total de categorías únicas, $y_i$ es el valor del atributo a predecir de una instancia  y $\mu = \frac{1}{N} \sum_{i=1}^{N}y_i$ es la media de todas esos valores en una división.\\
\subsection{Poda de árboles y criterios de terminación temprana.}
Una de las principales cuestiones a la hora de generar un árbol de decisión, en el que se intentan obtener los mejores resultados, es conocer cuál ha de ser el tamaño apropiado del mismo, ya que este factor va a influir considerablemente en la calidad de las predicciones proporcionadas. Un árbol muy pequeño corre el riesgo de haber generalizado más información de la cuenta, mientras que, un árbol muy grande, puede estar demasiado especializado, dejándose influir por ruido presente en las muestras y, como consecuencia, llegar a producir la situación denominada como sobreajuste, en la que, el árbol no ha sido capaz de extrapolar los datos relevantes para la clasificación y, por ello, podría colocar muestras en clases incorrectas.\\

Para evitar este tipo de problemas, es común recurrir a técnicas de podado de árboles. Podemos distinguir dos tipos de técnicas de poda:\\

\begin{itemize}
	\item Técnicas de poda realizadas antes de que se termine de generar el árbol \textit{(pre-pruning)}.
	\item Técnicas de poda realizadas tras la construcción del árbol \textit{(post-pruning)}.\\
\end{itemize}

Las técnicas de poda \textit{pre-pruning} ayudan también a que la construcción del árbol finalice antes y, de entre ellas, destacamos:\\

\begin{itemize}
	\item Establecer un mínimo de elementos por nodo/partición, de manera que cuando se alcanza dicho umbral esa partición no sigue siendo evaluada.
	\item Establecer una profundidad máxima del árbol.
	\item Establecer algún criterio de ganancia de información mínima.\\
\end{itemize}

En el momento en que una de estas condiciones se cumple, dicho nodo se convierte en un nodo terminal. En el caso de los problemas de clasificación, es común realizar el voto mayoritario, en el que se etiqueta una muestra con la clase más representativa del nodo, es decir la que tiene más instancias en el mismo. Por otro lado, en problemas de regresión es habitual etiquetar el nodo con la media de los valores a predecir ($\mu$) por el mismo.\\

De entre las técnicas de poda \textit{post-pruning} destacan dos:\\
\begin{itemize}
\item La \textbf{poda de error reducido}. Esta poda utiliza una técnica simple y rápida de computar en la que, empezando por cada una de las hojas del árbol, se va sustituyendo cada nodo por la clase más popular. Si la predicción no ha empeorado, se continúa en los niveles de profundidad anteriores al nivel actualmente podado y, en el momento en la que dicha predicción empeora, el procedimiento termina.\\

\item La \textbf{poda de coste-complejidad}. En la poda de coste-complejidad se genera una serie de árboles $T_0, T_1, T_i, ... , T_r$ donde $T_0$ es el árbol inicial y $T_r$ es sólo la raíz. En cada iteración ($i$) del proceso, se elimina un subárbol del árbol anterior ($i-1$) reemplazándolo con un nodo terminal conforme al siguiente criterio:

$error(T, S)$ es el error del árbol $T$ sobre el conjunto de datos $S$, que viene dado por el número de muestras cuya clase de salida original no se corresponde a la que el árbol que la evalúa predice.\\ $poda(T, t)$ es el árbol obtenido de podar el subárbol $t$ del árbol $T$.\\

En cada iteración, se elimina el subárbol que minimiza la diferencia entre el error obtenido de la operación aplicada al subárbol $t$ y el error al aplicarlo al conjunto total de datos, normalizado respecto de la diferencia entre el número de hojas del árbol antes y después de la poda realizada, respectivamente.

$$
\frac{error(poda(T, t), S) - error(T, S)}{|hojas(T)| - |hojas(poda(T, t))|}
$$

Una vez generados todos los árboles $T_0$ a $T_r$ se selecciona aquel que proporciona una mayor precisión.\\

\end{itemize}
Generalmente, las técnicas de poda \textit{post-pruning} suelen dar mejores resultados pero son más costosas computacionalmente.
\subsection{Calidad del modelo.}
La principal medida de la calidad del modelo generado aparte de su tiempo de ejecución es la capacidad que tiene de predecir la clase correcta para una muestra. A dicha medida se le denomina \textbf{precisión}.

$$
Precision = \frac{1}{N}\sum_{i=1}{N}[1|f(x_i) = y_i]
$$

donde $f$ es la función que nos devuelve la clasificación proporcionada por el árbol de decisión, $x_i$ es cada una de las muestras a evaluar e $y_i$ es su correcta clasificación. Para evaluar la precisión del modelo, se utiliza un conjunto reducido de datos, al que normalmente se denomina conjunto de test, que no ha sido utilizado durante el proceso de entrenamiento y del que conocemos su clase de salida de antemano.

\subsection{Random forest.}
Como explicaremos más adelante, para combinar nuestra implementación del árbol decisión en \textit{CUDA} y el framework de computación en clúster \textit{Spark}, optamos por realizar una implementación de un \textit{random forest} en vez de un único árbol de decisión. \\

Un \textit{\textbf{random forest}} es un conjunto de árboles de decisión, en el que el conjunto de muestras de entrenamiento es distribuido aleatoriamente en un número de árboles determinado por el usuario y, en el que, cada árbol del \textit{random forest}, es entrenado conforme a un árbol de decisión normal. Sin embargo, la evaluación de la muestra, con el fin de obtener la clase a la que pertenece, se realiza evaluando la muestra en cada uno de los árboles que componen el \textit{random forest} y tomando, como salida, la clase que un mayor número de árboles han predicho. \\

De esta manera, tenemos un modelo que, de forma similar al árbol de decisión, puede resolver problemas de clasificación o regresión no lineales, no requiere un gran preprocesamiento de las muestras antes de ser entrado para dar buenos resultados y, además, ayuda a evitar el sobreajuste, ya que, al distribuir aleatoriamente las muestras y tomar un promedio de los resultados, ayuda a impedir que el proceso de entrenamiento se especialice en exceso hacia muestra ruidosas. Sin embargo, como ocurría en el árbol, encontrar un \textit{random forest} óptimo global sigue siendo un problema NP-completo y, el modelo obtenido no es tan fácil de interpretar como el conjunto de reglas jerárquicas que presentaba el árbol de decisión.

%\underline{Ventajas.}\\
%\begin{itemize}
%	\item Proporcionan un modelo de caja blanca, fácil de interpretar y comprender.
%	\item Puede ser combinado con otros modelos distintos o en un conjunto de árboles (\textit{random forests}).
%	\item No requieren de un gran preprocesamiento de las muestras antes de ser entrenados.
%	\item A diferencia de otros modelos (como la regresión lineal), pueden resolver problemas de clasificación o regresión no lineales.\\
%\end{itemize}

%\underline{Inconvenientes.}\\
%\begin{itemize}
%	\item Pequeños cambios en el conjunto de datos utilizado para el entrenamiento pueden alterar considerablemente los resultados obtenidos.
	%\item Pueden proporcionar peores resultados que otros modelos no lineales (SVM, redes neuronales, etc) con los mismos datos pero, juntarlos en un conjunto con múltiples árboles, como los \textit{random forest} \cite{randomforest}, puede ayudarnos a solventar el problema, eso sí, eliminando la facilidad para interpretar el modelo que proporciona utilizar un único árbol.
	%\item Encontrar el árbol de decisión óptimo global es un problema NP-completo.
%\end{itemize}
%!TEX root = ../proyecto.tex
\chapter{Implementación.}
%\section{Breve introducción a CUDA.}
Como comentábamos al principio, CUDA \textit{(Computer Unified Device Arquitecture)} \cite{cuda} es una tecnología propietaria desarrollada por \textit{NVIDIA} y lanzada en junio de 2007, que nos proporciona de un lenguaje de programación general destinado a ser ejecutado en las tarjetas gráficas de la compañia. Para los propósitos de este trabajo y, habitualmente, a la hora de trabajar con CUDA denominaremos como \textbf{\textit{host}} a la CPU que se comunica con la tarjeta gráfica y como \textbf{dispositivo} a la GPU o tarjeta gráfica utilizada. \\

La intercomunicación entre \textit{host} y dispositivo sigue un modelo maestro-esclavo. El \textit{host} actúa como maestro y es el encargado de indicar al dispoistivo el código que ha de ejecutar y de mandarlo a la cola del dispositivo. Además, el \textit{host} tiene la posibilidad de trabajar de forma asíncrona con la GPU mientras la cola de trabajos del dispositivo no esté llena. \\

Es de vital importancia a la hora de trabajar con la GPU de tener en cuenta que:\\
\begin{itemize}
    \item a) La GPU tiene muchos más núcleos \textit{(cores)} que una CPU, lo que nos permite realizar mucha más operaciones en el mismo instante. Sin embargo, esto viene a expensas de un menor número de operaciones por segundo de cada núcleo, ya que para distrutar de la cantidad masiva de núcleos que tiene una GPU es necesario que ésta opere a una frecuencia más baja.

    \item b) La GPU tiene su propia estructura de memoria, que ha de usar para poder realizar operaciones. Dentro de la jerarquía de memoria encontramos memoria RAM similar a la que utiliza la CPU a través de la placa base, así como varios niveles de caché. Además, hemos de tener en cuenta que a la hora de ejecutar algo en la GPU vamos a tener un gasto extra de tiempo por el traspaso de información de CPU a GPU y viceversa. Minimizar la información que ha de traspasarse en ambos sentidos así como intentar que toda la información necesaria sea transferida a la vez para sacar máximo potencial del PCI Express y exprimir al máximo posible el uso eficiente de la memoria caché, que en CUDA es habitualmente realizado mediante el manejo de la ``memoria compartida'' es fundamental para obtener mejores resultados, especialmente, aquellos en los que el cuello de botella es la transferencia de datos.

    \item c) Como la GPU tiene su propia memoria dedicada de un tamaño limitado hemos de hacer hincapié en no utilizar soluciones que generan demasiada complejidad espacial, ya que limitan la escalabilidad de los algoritmos.
\end{itemize}

\subsection{Estructura de hebras, bloques y mallas.}
El \textbf{\textit{kernel}} es un fragmento de código especial destino a ser ejecutado en el dispotivo en el que se indica lo que ha de hacer una hebra.\\

Las \textbf{hebras} son la unidad mínima en la arquitectura CUDA. Cada hebra es ejecutada por un núcleo CUDA. Cada hebra es consciente en tiempo de ejecución de su identificador dentro del bloque así como del identificador del bloque en el que se encuentra y el tamaño del mismo, permitiéndonos así repartir el trabajo en función de dichos valores. El \textbf{bloque} se corresponde a un conjunto de hebras que ejecuta el mismo \textit{kernel} y que pueden cooperar entre sí y, al conjunto de esos bloques, se le denomina \textbf{``grid'' o malla}. Tanto las hebras dentro de un bloque como los bloques dentro de una malla puede tener estructuras unidimensionales, bidimensionales y tridimensionales. Las dimensiones de estas estructuras será indicada por el \textit{host} a la hora de ejecutar el \textit{kernel}.\\

CUDA exige que un mínimo de 32 hebras, denominado \textit{warp}, ejecuten instrucciones a la vez, aunque se hagan cálculos innecesarios así como que todas las hebras de un bloque sean ejecutadas por el mismo \textit{Streaming MultiProcessor}, de ahora en adelante, SM, que es uno de los procesadores en el dispositivo y dispone de un número específico de núcleos CUDA, sus propios registros y su propia caché entre otros.\\

Al lanzar un \textit{kernel} hemos de utilizar al menos un bloque de $N$ hebras. Además, en los casos unidimensionales el número de hebras por bloque está limitado a un máximo que depende de la tarjeta gráfica en cuestión.

\subsection{La memoria compartida.}
Dentro de la tarjeta gráfica, nos encontramos con distintos niveles de memoria. Una vez los datos necesarios han sido traspasados del \textit{host} al dispositivo a través del bus PCI Express, esos datos son almacenados en una memoria DRAM de propósito general del dispositivo. Cuando un \textit{kernel} solicita datos de esta memoria, de manera similar a como ocurre en una CPU, los datos solicitados y los colidantes en memoria son colocados a través de varios niveles de caché, que tiene tamaño más limitado que la memoria DRAM pero con un acceso de lectura y escritura mucho más rápido.\\

La \textbf{memoria compartida} es una abstracción para una región especial de la caché asociada a un bloque que es explícitamente usada por el programador en el \textit{kernel}, agilizando así considerablemente las transferencias de memoria en el dispositivo. En el cuadro \ref{tab:cudamemory}, podemos ver un resumen de los tipos de memoria existentes, dónde se pueden usar y dónde se encuentran dichos datos en el dispositivo.

\begin{table}[ht]
\begin{tabular}{|c|c|c|c|}
\hline
\textbf{Memoria}    & \textbf{Localización}                                           & \textbf{\begin{tabular}[c]{@{}c@{}}Acceso\\ (E = Escribir)\\ (L = Leer)\end{tabular}} & \textbf{\begin{tabular}[c]{@{}c@{}}Existente\\ hasta fin\\ de\end{tabular}} \\ \hline
\textbf{Registro}   & Caché                                                           & Kernel (E/L)                                                                          & Hebra                                                                       \\ \hline
\textbf{Local}      & \begin{tabular}[c]{@{}c@{}}DRAM\\ (Caché tras uso)\end{tabular} & Kernel (E/L)                                                                          & Hebra                                                                       \\ \hline
\textbf{Compartida} & Caché                                                           & Kernel (E/L)                                                                          & Bloque                                                                      \\ \hline
\textbf{Global}     & \begin{tabular}[c]{@{}c@{}}DRAM\\ (Caché tras uso)\end{tabular} & \begin{tabular}[c]{@{}c@{}}Host (E/L)\\ Kernel (E/L)\end{tabular}                     & \begin{tabular}[c]{@{}c@{}}Aplicación\\ o uso de free\end{tabular}          \\ \hline
\textbf{Constante}  & \begin{tabular}[c]{@{}c@{}}DRAM\\ (Caché tras uso)\end{tabular} & \begin{tabular}[c]{@{}c@{}}Host (E/L)\\ Kernel (L)\end{tabular}                       & \begin{tabular}[c]{@{}c@{}}Aplicación \\ o uso de free\end{tabular}         \\ \hline
\end{tabular}
\caption{Resumen de los tipos de memoria en CUDA.}
\label{tab:cudamemory}
\end{table}

\subsection{Python: Numba y CuPy.}
Para desarrollar el código asociado a este proyecto, hemos optado por utilizar \textbf{Python} en vez de los tradicionales C o C++. El uso de \textit{Python} nos permite un desarrollo de los algoritmos más rápido así como el acceso a abstracciones de más alto nivel mediante el uso de la librerías \textbf{\textit{Numba}} y \textbf{\textit{CuPy}},  así como una mayor facilidad para la distribución del código, si se desea, mediante el uso de \textit{PyPI(Python Package Index)}, el repositorio de paquetes para Python. \\

\textbf{Numba} \cite{numba} es un paquete para Python cuyo objetivo es la aceleración compilado fragmentos de código utilizando el compilador LLVM y dando la oportunidad de paralelizar código tanto para la CPU como para la GPU. En concreto, para las GPUs CUDA, proporciona al usuario un subconjunto de las características de CUDA con un nivel de abstracción mayor. Con eso no sólo conseguimos poder trabajar con CUDA desde Python sino, también evitar, si lo deseamos, manejar los traspasos de memoria entre host y dispositivo o la necesidad de indicar todos los tipos a la hora de inicializar un \textit{kernel} entre otras ventajas.
\begin{code}
\begin{minted}[fontsize=\footnotesize]{python}
from numba import cuda
import numpy as np
# Definimos el kernel
@cuda.jit
def aumentar_en_1(un_array):
  # Cogemos el índice de la hebra
    pos = cuda.grid(1)

    # Si el índice está en el rango del array
    # incrementamos su valor
    if pos < un_array.size:
        un_array[pos] += 1

if __name__ == '__main__':
  # Declaramos un array de 10000 ceros
  ejemplo = np.zeros(10000)
  # Calculamos el número de bloques necesario
  bloques = ejemplo.size // 128 + 1
  # Lanzamos el kernel con bloques de 128 hebras
  aumentar_en_1[bloques, 128](ejemplo)
\end{minted}
\captionof{listing}{Kernel para incrementar en 1 los elementos de un array.\\\\}
\label{code:numbaexample}
\end{code}

\textbf{CuPy} \cite{cupy} es otro paquete de Python que, por un lado y de manera similar a Numba, nos permite generar kernels para CUDA en este caso de manera similar a los de C/C++ así como facilidades para generar kernels en los que se implementa reducciones u operaciones elemento a elemento en un array. Por otro lado, proporciona una API similar a la de NumPy pero las operaciones están implementadas utilizando CUDA. Además, CuPy está implementado de manera que permite utilizar directamente sus estructuras de datos sobre kernels de Numba, lo que nos permite combinar elementos de ambos paquetes según nos interese.

\subsection{Spark.}
\textit{Apache Spark} es un \textit{framework} de código abierto y propósito general para sistemas distribuidos de computación en clúster que proporciona una API utilizable desde los lenguajes de programación en Scala, Java, Python y R. El \textit{framework} fundamenta su arquitectura en el \textit{RDD (Resilient Distributed DataSet)}, que es una estructura de datos de sólo lectura distribuida en un clúster de máquinas, mantenida durante toda la computación y con tolerancia a fallos. Además, proporciona otras herramientas de alto nivel como ML/MLib, una librería con algoritmos de \textit{machine learning}.\\

Utilizando la API de Python, podemos combinar el uso de \textit{Spark} y \textit{Numba CUDA} para afrontar problemas de grandes dimensiones, ya que el \textit{RDD} nos permite trabajar con subconjuntos de esos datos posibilitando incluso llevar las implementaciones realizadas a un clúster con múltiples sistemas con dispositivos GPU \textit{CUDA} cont todas las dependencias necesarias instaladas. \\

La distribución de trabajo en Spark se realizará utilizando la transformación \textit{mapPartitions} del \textit{RDD} de \textit{Spark}, que generá un nuevo RDD a partir de los resultados obtenidos al aplicar la función pasada a \textit{mapPartitions} como parámetro a cada una de las funciones.


\section{Proceso de implementación.}
Para realizar la implementación de cada algoritmo hemos realizado un proceso cíclico divido en 3 fases:

\begin{itemize}
	\item \textbf{Análisis} - En la primera iteración, analizar los trabajos relacionados. En las posteriores, analizar los resultados obtenidos del profiler, determinar los cuellos de botella y buscar posibles alternativas para solucionar el problema.
	\item \textbf{Implementación} - Realizar la implementación en CUDA de los cambios o elementos nuevos obtenidos del proceso de análisis.
	\item \textbf{Profiling} - Utilizar el profiler de NVIDIA, \textit{Nsight}, sobre un ejemplo razonable para evaluar el rendimiento del algoritmo.
\end{itemize}

\section{Desarrollo del mapa auto-organizado de Kohonen.}
Para implementar los mapas auto-organizados de Kohonen, primero consideramos la versión tradicional \textit{online} y, posteriormente, tras ver las limitaciones de la primera, evaluamos la versión computada en \textit{batchs}, que ha sido implementada tanto para CPU, usando \textit{NumPy}, como para CUDA, usando \textit{Numba}.

\subsection{Limitaciones del mapa auto-organizado online.}
Mientras que la implementación del mapa auto-organizado \textit{online} fue el punto de partida para la realización de este trabajo, tuvimos que descartar esta versión del algoritmo, ya que, el objetivo de este trabajo es resolver problemas con un gran número de muestras utilizando \textit{CUDA} y \textit{Spark}.\\

En esta versión, en cada iteración, se selecciona una única muestra del conjunto de datos y ésta es evaluada para actualizar los pesos de las neuronas, que serán el punto de partida de la siguiente iteración, limitando a una el número de muestras que pueden procesarse a la vez y, por tanto, secuencializando el proceso.\\

La opción más apropiada para paralelizar esta versión del algoritmo sería procesar una única muestra usando tantas hebras como neuronas tenga el mapa de salida. En ese caso, en cada iteración, cada hebra podría calcular su distancia euclídea de la muestra con los pesos de la neurona asociada a la hebra, usaríamos el algoritmo de la reducción, que explicaremos posteriormente, para encontrar la BMU y, cada hebra, realizaría la actualización de los pesos de su neurona, si ese fuera el caso. Sin embargo, este procedimiento sólo conseguiría ganancias significativas con respecto a su versión para CPU con un mapa de neuronas considerablemente grande, factor que no parece razonable en un algoritmo cuyo principal uso es aprovechar las ventajas que proporciona el \textit{clustering} para procesar cantidades muy grandes de datos.\\

Determinadas estas limitaciones y, dado que nuestro objetivo es evaluar un conjunto con un número de muestras elevado, optamos por implementar la versión del mapa auto-organizado que nos permite evaluar múltiples muestras simultáneamente, conocido como el mapa auto-organizado \textit{batch}.\\


Para el desarrollo de esta versión del algoritmo hemos combinado el uso de \textit{CUDA} mediante \textit{Numba} y \textit{Spark}. En primer lugar, vamos a ver un esquema general del uso de \textit{Spark} para afrontar nuestro algoritmo iterativo y, a continuación, explicaremos en detalle la implementación de los \textit{kernels} para \textit{CUDA}.

\subsection{Uso de Spark.}
\begin{figure}[ht]
\centering
\includegraphics[scale=0.25]{imagenes/flujosparksom.png}
\caption{Diagrama de flujo del mapa auto-organizado desarrollado.}
\label{image:flujosparksom}
\end{figure}

Utilizar \textit{Spark} para implementar este algoritmo nos permite afrontar problemas de tamaños superiores a la capacidad de memoria de nuestro dispositivo, siempre que la memoria necesaria para evaluar una partición del \textit{RDD} quepa en la memoria del dispositivo, y llevar la implementación realizada a un clúster con múltiples nodos, si cada nodo tiene acceso a un dispositivo \textit{CUDA} con todas las dependencias de \textit{software} instaladas.\\


El algoritmo comienza con un  único nodo de \textit{Spark}, utilizando el primer kernel desarrollado, \textit{rand\_weights}, para de esta forma inicializar de manera pseudoaleatoria los valores de la estructura de pesos, en función de una semilla proporcionada por el usuario. Con esta estructura ya generada, empieza el proceso iterativo en el que:
\begin{enumerate}
    \item Calculamos el parámetro de control $\sigma$ para la iteración en función de las ecuaciones correspondientes.

$$\sigma(t) = \left\{
\begin{array}{ll}
\sigma_0e^{-\frac{t}{\tau}} & si \;\;t < z\\
\sigma_f & si  \;\; t\geq z
\end{array}
\right.
$$\\
    \item Utilizando la transformación \textit{mapPartitions}, en cada partición del \textit{RDD} se aplica la función \textit{gpu\_work\_iter}, que encapsula el segundo  \textit{kernel} desarrollado, \textit{som\_iter}. Este \textit{kernel} se encarga de evaluar los pesos parciales para cada neurona en función de las muestras asociadas a la partición del \textit{RDD}. Puesto que la actualización de pesos es una división entre una sumatoria de vectores, con cada vector del tamaño de una muestra, y una sumatoria de números reales, el objetivo de cada partición será calcular esos numeradores y denominadores, a los que nos referiremos de ahora en adelante como numeradores y denominadores parciales.
    $$
 W_{i, j} = \frac{\sum_{k=0}^{f} \delta_f(c, [i,j]) \cdot  X(T_k) }{\sum_{k=0}^{f} \delta_f(c, [i,j])}
$$
    \item Para finalizar la iteración, \textit{Spark} reúne mediante \textit{collect} las numeradores y denominadores parciales obtenidos y, usando el último \textit{kernel} implementado, \textit{finish\_update} obtiene los pesos finales de la iteración.\\
\end{enumerate}

Este proceso, que consta de 3 fases, es realizado hasta alcanzar el número máximo de iteraciones. Hemos de destacar que todas las particiones han de partir de la mismos pesos en cada iteración para realizar los cálculos. Por ello, al inicio de la iteración, es necesario distribuir la matriz de pesos a cada nodo de \textit{Spark} que realiza esos cálculos y, al final de la iteración, reunir todos los numeradores y denominadores parciales en un único nodo, permitiéndonos obtener los pesos finales de la iteración. \\

Para que \textit{Spark} pueda realizar esa distribución a lo largo de un clúster es necesario que, al final de la iteración, se haga la transferencia desde la memoria de dispositivo al \textit{host} ,de los pesos parciales y, al inicio de la iteración, se haga la transferencia desde el \textit{host} al dispositivo de la pesos de las neuronas correspondientes a esa iteración.\\

\begin{code}
\begin{minted}[fontsize=\footnotesize]{python}

def spark_gpu_batch_som(rdd_data, d, max_iters, rows, cols, smooth_iters=None,
                        sigma_0=10, sigma_f=0.1, tau=400, seed=None, tpb=128):
    """
    :param rdd_data RDD con el conjunto de muestras a evaluar.
    :param d Tamaño de una muestra, dimensión del problema.
    :param max_iters Número de iteraciones a realizar.
    :param rows Número de filas en el mapa de neuronas.
    :param cols Número de columnas en el mapa de neuronas.
    :param smooth_iters Número de iteraciones en las que el parámetro
           sigma decrece siguiendo una función gaussiana. 
    :param sigma_0 Valor de sigma inicial.
    :param sigma_f Valor de sigma tras alcanzar la iteración smooth_iters.
    :param tau Valor de tau para la función gaussiana.
    :param seed Semilla pseudoaleatoria para la generación inicial de pesos.
    :param tpb Número de hebras por bloque para la inicialización de pesos y
           la actualización final de los pesos.
    """
    # 1. Declaramos la estructura de los pesos.
    d_weights = cuda.device_array((rows, cols ,d), np.float32)

    # 1.2 Usamos Numba para generar los pesos de forma pseudoaleatoria.
    rng_states = create_xoroshiro128p_states(rows * cols * d, seed=seed)
    rand_weights[(d_weights.size) // tpb + 1, tpb](rng_states, d_weights)
     
    # 1.3 Traemos los pesos de la memoria de la GPU a la memoria del host.
    weights = d_weights.copy_to_host()

    # 2. Inicio del proceso iterativo
    for t in range(max_iters):
        # 2.a Actualizamos sigma en función de los tau y la iteración.
        if smooth_iters is None or t < max_iters:
            sigma = sigma_0 * math.exp((-t/tau))
        else:
            sigma = sigma_f
            
        sigma_squared = sigma * sigma
        
        # 2.b Cálculos parciales con mapPartitions en cada nodo.
        out = rdd_data.mapPartitions(gpu_work_iter(weights, sigma_squared))
        
        # 2.c En un único nodo calculamos las sumas parciales.
        out = out.collect()
        finish_update[rows*cols//tpb + 1, tpb](weights, np.concatenate(out), 
                                               len(out) // 2)
       
    # 3. Devolvemos los pesos obtenidos
    return weights
\end{minted}
\captionof{listing}{Uso de Spark para entrenar el mapa auto-organizado.}
\label{code:somspark}
\end{code}

\subsection{Representación de la estructura de pesos de las neuronas.}
La estructura que contiene los pesos de las neuronas, que durante la ejecución de los \textit{kernels} se encontrará almacenada en la memoria global del dispositivo, se corresponde a un array tridimensional. El primer eje indica la fila que ocupa la neurona en el mapa, el segundo eje indica la columna que ocupa la neurona en el mapa y el último eje la característica del problema a la que queremos acceder.\\

Mientras que nosotros podemos hacer uso de este sistema de indexación tridimensional gracias a Numba, en realidad, en el dispositivo CUDA se trata de un array unidimensional \textit{row-major}.


\begin{figure}[ht]
\centering
\includegraphics[scale=2.0]{imagenes/row-major.png}
\caption{Representación de un array 3D como un array 1D row-major.}
\label{image:rowmajor}
\end{figure}

\subsection{Kernels implementados.}
\subsubsection{Generación pseudoaleatoria de pesos de neuronas.}
\begin{code}
\begin{minted}[fontsize=\footnotesize]{python}
@cuda.jit
def rand_weights(rng_states, d_weights):
    """
    Kernel para inicializar aleatoriamente la estructura de pesos con 
    valores en el intervalo [0, 1) tomados de una distribución uniforme
    :param rng_states Estados aleatorios.
    :param d_weigths Vector de filas * columnas * d valores que contendrá 
           los pesos asociados a las neuronas.
    """
    # La hebra coge su identificador unidimensional único.
    idx = cuda.grid(1)

    # La hebra calcula en función del su índice 
    # y cocientes y restos de divisiones entereas
    n_rows, n_cols, d = d_weights.shape

    # Cálculo de la fila (eje X).
    row = idx // (n_cols * d)

    # Cálculo de la columna (eje Y).
    col_d = idx % (n_cols * d)
    col = col_d // d
    # Cálculo de la característica (eje Z).
    i = col_d % d
    
    # Sacamos el aleatorio correspondiente.
    if idx < d_weights.size:
        d_weights[row, col, i] = xoroshiro128p_uniform_float32(rng_states, idx)

\end{minted}
\captionof{listing}{Inicialización pseudoaleatoria de los pesos de las neuronas.\\}
\label{code:numbainitweights}
\end{code}

Este proceso (código fuente \ref{code:numbainitweights}) es realizado por el nodo de Spark que controlaría la ejecución del clúster una única vez al inicio del algoritmo, pero utilizando la GPU. \textit{Numba CUDA} nos proporciona herramientas para la generación de valores flotantes en el rango comprendido entre 0 y 1 basadas en el método de Box-Muller. Hemos utilizado esta herramienta para la generación de nuestra matriz de vectores de pesos inicial. Una vez generados, son trasladados de vuelta a la CPU para ser distribuidos a todos los nodos ejecutores de \textit{Spark}. \\

Al lanzar este kernel, se utilizan tantas hebras como números aleatorios (tabla \ref{tab:randkernel}), distribuidos en bloques de un tamaño indicado por el usuario.
\begin{table}[ht]
\begin{tabular}{@{}lll@{}}
\toprule
\textbf{Kernel}        & \textbf{Bloques}                                 & \textbf{Hebras por bloque}                                                                       \\ \midrule
\textbf{rand\_weights} & ($filas$ $\cdot$ $columnas$ $\cdot$ $d$) $//$ $tpb$ + 1 & $tpb$ \\ \bottomrule
\end{tabular}

\textit{\\$d=$ dimensión del problema.\\$tpb=$ hebras por bloque (indicados por el usuario).\\ $//=$ cociente de división entera.}
\caption{Parámetros para el lanzamiento del kernel rand\_weights.}
\label{tab:randkernel}
\end{table}
\newpage
\subsubsection{Cálculo de los numeradores y denominadores parciales.}
\begin{code}
\begin{minted}[fontsize=\footnotesize]{python}
def gpu_work_iter(weights, sigma_squared):
    def _gpu_work(data):
        # 1. Procesamos el dataset
        inp = np.asarray(list(data), dtype=np.float32)
        rows, cols, d = weights.shape
        nneurons = rows * cols
        
        # 2. Pasamos los datos a las memorias del dispositivo
        d_samples = cuda.to_device(inp)
        d_weights = cuda.to_device(weights)
        nums = np.zeros(rows * cols * d, np.float32)
        denums = np.zeros(rows * cols, np.float32)
        d_nums = cuda.to_device(nums)
        d_denums = cuda.to_device(denums)
        
        # 3. Tomamos el número de hebras por bloque
        if nneurons > 1024:
            raise Exception('Número de neuronas superior al límite')
        # Número de hebras necesario para que funcione la reducción.
        tpb = max(64,2**(math.ceil(math.log2(nneurons))))
        # 4. Lanzamos el kernel.
        # Memoria compartida para almacenar una muestra por bloque
        sm_size = 4 * d
        som_iter[N, tpb, 0, sm_size](d_samples, d_weights, d_nums, d_denums,
                                     sigma_squared)
        
        return d_nums.copy_to_host(), d_denums.copy_to_host()
    return _gpu_work
\end{minted}
\captionof{listing}{Función a ejecutar con mapPartitions.\\}
\label{code:somencapsulado}
\end{code}
Una vez obtenidos los pesos iniciales de una iteración, el siguiente paso es utilizar \textit{mapPartitions} para obtener los pesos parciales de cada partición del \textit{RDD}, como veíamos en el código fuente \ref{code:somspark}. La función utilizada en cada partición se denomina \textit{gpu\_som\_iter} y encapsula el lanzamiento del kernel \textit{som\_iter} y las transferencias de memoria entre host y dispositivo en cada iteración.\\

\begin{table}[ht]
\centering
\begin{tabular}{@{}lll@{}}

\toprule
\textbf{Kernel}        & \textbf{Bloques}                                 & \textbf{Hebras por bloque}                                                                       \\ \midrule
\textbf{som\_iter} &   Nº de muestras. & $max(64,2^{techo(\log_2{Nº neuronas})})$ \\ \bottomrule
\end{tabular}
\textit{\\techo=menor número entero mayor o igual que un número real.}
\caption{Parámetros para el lanzamiento del kernel som\_iter.}
\label{tab:iterkernel}
\end{table}

El \textit{kernel som\_iter} es la parte más importante de la implementación del algoritmo y realiza todas las operaciones necesarias para obtener los numeradores y denominadores parciales de la iteración.



\begin{code}
\begin{minted}[fontsize=\footnotesize]{python}
@cuda.jit
def som_iter(d_samples, d_weights, d_nums, d_denums, sigma_squared):
    """
    :param d_samples Conjunto de todas las muestras a evaluar.
    :param d_weigths Vector de filas * columnas * d valores que contendrá 
           los pesos asociados a las neuronas.
    :param d_nums Vector con los numeradores para el cálculo de la fórmula.
    :param d_denums Vector con los denominadores para el cálculo de la fórmula.
    :param sigma_squared Valor de sigma al cuadrado para el cáculo del vecindario.
    """
    nrows, ncols, d = d_weights.shape
    nneurons = nrows * ncols
    
    sample_idx, nueron_idx = cuda.blockIdx.x, cuda.threadIdx.x
    neuron_row, neuron_col = neuron_idx // ncols, neuron_idx % ncols
    blockSize = cuda.blockDim.x
       
    # 0. Declaramos la memoria compartida
    shared_sample = cuda.shared.array(shape=0, dtype=numba.float32)
    shared_distances = cuda.shared.array(shape=1024, dtype=numba.float32)
    shared_idx = cuda.shared.array(shape=1024, dtype=numba.int32)
    
    # 1.a Cada hebra pone una posición de la muestra en memoria compartida.
    # El bucle for permite realizar esto si la dimensión del problema fuese
    # superior al número de neuronas.
    for i in range(d // nneurons + 1):
        i_stride = i * nneurons
        my_pos = i_stride + cuda.threadIdx.x
        # Si la posición que corresponde a la hebra no supera el
        # tamaño de la muestra a cargar.
        if my_pos < d: 
            shared_sample[my_pos] = d_samples[sample_idx, my_pos]
    # Sincronizamos para asegurar que la muestra ha sido cargada.
    cuda.syncthreads()
    
    # 1.b Calculamos las distancias euclídeas que nos corresponden.
    if neuron_idx < nneurons:
        shared_distances[neuron_idx] = 0.0
        for i in range(d):
            i_distance = shared_sample[i] - d_weights[neuron_row, neuron_col, i]
            shared_distances[neuron_idx] += i_distance * i_distance
    # Si hay más hebras que neuronas inicializamos a infinito para la reducción.
    else: 
        shared_distances[neuron_idx] = np.inf
    
    # 1.c Inicializamos el array de índices para la reducción.
    shared_idx[neuron_idx] = neuron_idx
    # Sincronizamos para asegurar los arrays han sido inicializados.
    cuda.syncthreads()    
\end{minted}
\captionof{listing}{Primer fragmento [Cálculo de distancias] de som\_iter.\\}
\label{code:somiter1}
\end{code}

El kernel comienza con la declaración e inicialización de la memoria compartida.\\

En primer lugar, cada hebra contribuye a cargar una característica de la muestra a evaluar por el bloque hasta la muestra ha sido cargada por completo. En segundo lugar, generamos dos arrays adicionales en memoria compartida, que serán utilizados posteriormente para calcular la BMU. Puesto que hemos limitado nuestra implementación a funcionar con un máximo de 1024 neuronas, que es el máximo de hebras por bloque, estos dos arrays serán siempre de esta dimensión. Uno de ellos, que será de \textit{floats} de 32 bits, contendrá las distancias entre la muestra que cargamos en memoria compartida y los pesos de cada neurona del mapa. El segundo, que será de enteros de 32 \textit{bits}, será inicializados con los índices de cada neurona. Para realizar el cálculo de la distancia euclídea, cada hebra calculará su distancia con la neurona que le corresponde y la muestra cargada en memoria compartida. Si hubiese más hebras en el bloque que neuronas en el mapa, el resto de distancias son inicializadas a infinito. \\

Puesto que nuestro siguiente objetivo será encontrar la BMU, es decir, la neurona con menor distancia, no es necesario calcular la raíz cuadrada de la división euclídea, ya que ésta no afecta a la relación de orden. Para encontrar la distancia mínima, utilizamos un algoritmo frecuentemente utilizando en la GPU: \textbf{la reducción}.
\begin{figure}[ht]
\centering
\includegraphics[scale=0.5]{imagenes/parallel_reduce.png}
\caption{Una reducción paralela de una sumatoria.}
\label{image:cudareduction}
\end{figure}

\begin{code}
\begin{minted}[fontsize=\footnotesize]{python}
    # Recorrido de árbol de hojas a la raíz (posición 0)
    if blockSize >= 1024 and neuron_idx < 512:
        if shared_distances[neuron_idx + 512] < shared_distances[neuron_idx]:
            shared_distances[neuron_idx] = shared_distances[neuron_idx + 512]
            shared_idx[neuron_idx] = shared_idx[neuron_idx + 512]
    cuda.syncthreads()
    
    if blockSize >= 512 and neuron_idx < 256:
        if shared_distances[neuron_idx + 256] < shared_distances[neuron_idx]:
            shared_distances[neuron_idx] = shared_distances[neuron_idx + 256]
            shared_idx[neuron_idx] = shared_idx[neuron_idx + 256]
    cuda.syncthreads()
    
    if blockSize >= 256 and neuron_idx < 128:
        if shared_distances[neuron_idx + 128] < shared_distances[neuron_idx]:
            shared_distances[neuron_idx] = shared_distances[neuron_idx + 128]
            shared_idx[neuron_idx] = shared_idx[neuron_idx + 128]
    cuda.syncthreads()
    
    if blockSize >= 128 and neuron_idx < 64:
        if shared_distances[neuron_idx + 64] < shared_distances[neuron_idx]:
            shared_distances[neuron_idx] = shared_distances[neuron_idx + 64]
            shared_idx[neuron_idx] = shared_idx[neuron_idx + 64]
    cuda.syncthreads()
    
    if neuron_idx < 32: # Unroll de warp. No necesitamos sincronizar.
        if shared_distances[neuron_idx + 32] < shared_distances[neuron_idx]:
            shared_distances[neuron_idx] = shared_distances[neuron_idx + 32]
            shared_idx[neuron_idx] = shared_idx[neuron_idx + 32]
        if shared_distances[neuron_idx + 16] < shared_distances[neuron_idx]:
            shared_distances[neuron_idx] = shared_distances[neuron_idx + 16]
            shared_idx[neuron_idx] = shared_idx[neuron_idx + 16]
        if shared_distances[neuron_idx + 8] < shared_distances[neuron_idx]:
            shared_distances[neuron_idx] = shared_distances[neuron_idx + 8]
            shared_idx[neuron_idx] = shared_idx[neuron_idx + 8]
        if shared_distances[neuron_idx + 4] < shared_distances[neuron_idx]:
            shared_distances[neuron_idx] = shared_distances[neuron_idx + 4]
            shared_idx[neuron_idx] = shared_idx[neuron_idx + 4]
        if shared_distances[neuron_idx + 2] < shared_distances[neuron_idx]:
            shared_distances[neuron_idx] = shared_distances[neuron_idx + 2]
            shared_idx[neuron_idx] = shared_idx[neuron_idx + 2]
        if shared_distances[neuron_idx + 1] < shared_distances[neuron_idx]:
            shared_distances[neuron_idx] = shared_distances[neuron_idx + 1]
            shared_idx[neuron_idx] = shared_idx[neuron_idx + 1]
    cuda.syncthreads()
    
    # La mejor neurona se encuentra en la posición 0 del array.
    bmu = shared_idx[0]
    bmu_row, bmu_col = bmu // ncols, bmu % ncols
    cuda.syncthreads()
\end{minted}
\captionof{listing}{Segundo fragmento [Reducción] del kernel som\_iter.\\}
\label{code:somiter2}
\end{code}
  
La reducción puede utilizarse para obtener el resultado de aplicar un operador binario a lo largo de un array, siempre que el operador en cuestión cumpla la propiedad asociativa. En nuestro caso, dicho operador es el mínimo entre dos elementos. Para realizar esta operación de manera eficiente dentro de un bloque, se simula un recorrido hacia arriba sobre un árbol binario balanceado (figura \ref{image:cudareduction}), en el que vamos aplicando la operación sobre los dos hijos y guardando el resultado en el nodo padre, tomando los distancias cargadas en memoria compartida como las hojas y alcanzado el resultado final en la raíz. Por ello, era necesario que el número de hebras por bloque fuese potencia de 2, si teníamos más hebras que neuronas completábamos las distancias con infinito, que actúa como elemento neutro de la operación mínimo, y, añadíamos un array extra con los índices para propagar la posición con el mejor índice mientras hacemos el recorrido. Podemos consultar con más detalle cómo realizar una implementación de una reducción de alto rendimiento en \textit{CUDA} en la referencia bibliográfica \cite{reduction}.\\


Para finalizar el \textit{kernel}, se realiza el cálculo de los numeradores y denominadores parciales. Para ello, cada hebra del bloque se corresponde con una neurona y mide la distancia euclídea que existe entre la posición de la BMU y la posición de la neurona en el mapa. Si esa distancia es menor o igual que el parámetro de control $\sigma^2$, se realiza la suma del vector del numerador con el producto de esa distancia y la muestra guardada en la memoria compartida del bloque y sólo la distancia con el denominador. \\
\begin{code}
\begin{minted}[fontsize=\footnotesize]{python}
# 3. Realizamos la actualización de los pesos.
    if neuron_idx < nneurons:
        dist = (neuron_row - bmu_row) * (neuron_row - bmu_row) + \
               (neuron_col - bmu_col) * (neuron_col - bmu_col)
        # Si estamos dentro del rango de actualización.
        if dist <= sigma_squared:
            hck = math.exp(-dist/(2 * sigma_squared))
            # Guardamos sumatoria del denominador.
            cuda.atomic.add(d_denums, neuron_row * ncols + neuron_col, hck)
            # Guardamos sumatoria del numerador.
            for i in range(d):
                cuda.atomic.add(d_nums, neuron_row*ncols*d + neuron_col*d+i,
                                hck * shared_sample[i])
\end{minted}
\captionof{listing}{Tercer y último fragmento del kernel som\_iter.\\}
\label{code:somiter3}
\end{code}


Puesto que múltiples hebras pueden tener la misma BMU y, por tanto, estar actualizando las mismas posiciones en memoria a la vez se utilizan \textbf{operaciones atómicas}, que evitan las condiciones de carrera que pueda surgir a cambio de una mayor latencia en la operación. Hemos de indicar que, para las operaciones atómicas, necesitamos trabajar con arrays unidimensionales, por lo que hemos de hacer los cálculos de indexación necesarios para acceder a las posiciones de memoria deseadas.

\subsubsection{Cálculo de los pesos finales de la iteración.}
Una vez todos los resultados han sido recopilados en un nodo de \textit{Spark}, lanzamos el \textit{kernel finish\_update}, que realizará la sumatoria de los numeradores parciales y de los denominadores parciales para cada neurona así como la división entre ambos. Si ninguna muestra activó la neurona en cuestión, es decir, la suma de todos sus denominadores parciales es 0, se mantendrán los pesos de la iteración anterior para esa neurona. En caso contrario,  los pesos de la neurona se corresponden con el vector obtenido de la división. Para lanzar este \textit{kernel} se utilizan tantas hebras como neuronas hay en el mapa, divididas en bloque de \textit{tpb} hebras. Cada hebra realiza los cálculos asociados a una neurona.

\begin{table}[ht]
\begin{tabular}{@{}lll@{}}
\toprule
\textbf{Kernel}        & \textbf{Bloques}                                 & \textbf{Hebras por bloque}                                                                       \\ \midrule
\textbf{finish\_update} & (Nº de neuronas // $tpb$ + 1) & $tpb$ \\ \bottomrule
\end{tabular}
\caption{Parámetros para el lanzamiento del kernel finish\_update.}
\label{tab:updatekernel}
\end{table}

\begin{code}
\begin{minted}[fontsize=\footnotesize]{python}
@cuda.jit
def finish_update(d_weights, partials, numParts):
    """
    :param d_weights Array de pesos de neuronas.
    :param partials Array con sumas parciales.
    :param numParts Número de resultados parciales a procesar.
    """
    idx = cuda.grid(1)
    nrows, ncols, d = d_weights.shape
    if idx < nrows * ncols:
        row, col = idx // ncols, col = idx % ncols
        
        # a) Sumamos todos los parciales en el primer array.
        numsize = nrows * ncols * d
        densize = nrows * ncols
        fullsize = numsize + densize
        for i in range(numParts - 1):
            # Suma de numeradores.
            for k in range(d):
                pos = fullsize * i + row * ncols * d + col * d + k
                partials[row * ncols * d + col * d + k] += partials[pos]
            # Suma de denominadores.
            pos = fullsize * i + numsize + row * ncols + col
            partials[numsize + row * ncols + col] += partials[pos]
    
        # b) Si no es 0 el denominador realizamos la división y cambiamos pesos.
        if partials[numsize + row * ncols + col] != 0:
            for k in range(d):
                d_weights[row, col, k] = partials[row*ncols*d + col*d +k] / \
                                         partials[numsize + row * ncols + col]
\end{minted}
\captionof{listing}{Actualización final de la matriz de pesos.\\}
\label{code:ending}
\end{code}

\section{Desarrollo de un modelo de árbol de decisión.}
La implementación del modelo de árbol de decisión se basa en CUDT \cite{cudt}, que a su vez se fundamenta en SPRINT \cite{sprint} y la operación de \textit{scan}.

\subsection{Lista de atributos.}
Una lista de atributos, es una estructura auxiliar, procedente de SPRINT \cite{sprint}, utilizada para representar las clases y los atributos asociados a una muestra. Una lista de atributos tiene una estructura similar a la siguiente tabla:

\begin{table}[ht]
\centering
\begin{tabular}{@{}lll@{}}
\toprule
Valor & Clase & ID Muestra \\ \midrule
2,5   & 0     & 0          \\
4,7   & 0     & 1          \\
0,1   & 1     & 2          \\
1,0   & 1     & 3          \\ \bottomrule
\end{tabular}
\caption{Una lista de atributos sin ordenar.}
\label{tab:ejlistaatributos}
\end{table}

Las columnas de la tabla \ref{tab:ejlistaatributos} son:
\begin{itemize}
    \item \textbf{Valor}, que se corresponde al valor que toma el atributo al que corresponde la tabla en la muestra representada en la fila.
    \item \textbf{Clase}, que se corresponde a la etiqueta de salida asociada a la muestra de la fila.
    \item \textbf{ID Muestra}, que se corresponde al identificador de la muestra. Al principio, se corresponde al número de fila empezando por 0.
\end{itemize}

Una vez esta estructura es generada para cada atributo del problema en cuestión, es ordenada por orden creciente según la columna ``Valor''. En la implementación realizada, se utiliza un array para cada columna.

\subsection{Esquema general del algoritmo implementado.}
\begin{figure}[ht]
\centering
\includegraphics[scale=0.35]{imagenes/esquemadtree.png}
\caption{Diagrama de flujo de la implementación del árbol de decisión.}
\label{img:dtree}
\end{figure}
Al inicio del algoritmo, tras generar las listas de atributos, se genera un nodo raíz que comprende todas las muestras del conjunto. En ese nodo, hemos de encontrar para qué atributo y qué valor realizamos la partición óptima de los datos. Para ello, se consideran todas las listas de atributos y se toma como posible punto de corte el punto medio entre un valor y el siguiente si no se trata del mismo valor. Asociada a cada una de las particiones, se calcula el criterio de Gini. Una vez realizados todos los cálculos, tomamos como punto de corte aquella que menor criterio de Gini nos de. Utilizando ese punto de corte, generamos dos nuevos nodos para la siguiente iteración, uno que contiene todos los puntos menores o iguales que dicho punto de corte y otro con los mayores. Además, dicho punto es el que utilizamos para generar nuestro nodo de decisión en el árbol entrenado.  Este proceso se repite hasta que no quedan nodos por evaluar. Un nodo no ha de ser evaluado si:\\

\begin{enumerate}[a.]
    \item Todos los elementos del nodo pertenecen a la misma clase. En ese caso, en vez de un nodo de decisión, generamos un nodo terminal con la clase correspondiente.
    \item Se ha especificado un criterio de profundidad máxima y dicha profundidad ha sido alcanzada. En ese caso, generamos un nodo terminal con la clase más representativa del nodo.
    \item Se ha especificado un límite para el número de elementos mínimo que puede contener y ha sido alcanzado. En ese caso, generamos un nodo terminal de manera similar al caso anterior.
\end{enumerate}



\subsection{La operación de scan.}
Una de las claves del uso de la lista de atributos, es que, para los problemas de \textbf{clasificación binaria}, que son los únicos que nuestro modelo es capaz de resolver, si codificamos una clase como $0$ (a partir de ahora llamada \textit{clase negativa}) y otra como $1$ (\textit{clase positiva}) si realizamos una suma acumulada sobre el subconjunto de filas de un nodo de la columna ``Clase'' podríamos tener control de cuántos elementos hay en cada clase tanto para todas las particiones. Para realizar la suma acumulada existe una primitiva ampliamente utilizada en el mundo de la GPU denominada \textit{scan}.\\

El \textit{scan} \cite{scan}, suma acumulada o suma prefija, es una operación que utiliza un operador binario, $\oplus$, que cumpla la propiedad asociativa y utilizada sobre un \textit{array} de $n$ elementos. Existen dos formas de realizar el \textit{scan}: inclusivo y exclusivo. El scan inclusivo empieza con el primer elemento del array y va a realizando una suma acumulada. El \textit{scan} exclusivo empieza con el elemento neutro de la operación y realiza una suma acumulada de todos los elementos hasta el penúltimo. En la implementación realizada, hemos utilizado el \textit{scan inclusivo}. Por tanto, para los propósitos de este documento, cada vez que hablemos de \textit{scan}, nos estaremos refiriendo al \textit{scan inclusivo}, que al aplicarla sobre un array, nos devuelve lo siguiente:\\
$$scan([a_0, a_1, a_2, ..., a_{n-1}]) = [ a_0, (a_0 \oplus a_1), (a_0 \oplus a_1 \oplus a_2), ..., (a_0 \oplus a_1 \oplus a_2 \oplus ... \oplus a_{n})]$$

La implementación realizada utilizada operaciones directas sobre \textit{warps}, permitiendo que una ejecución más rápida al realizar todas las operaciones directamente sobre los registros del multiprocesador. De manera más específica el procedimiento realizado es el siguiente.\\

\begin{enumerate}
    \item Se lanza un kernel con una estructura con un número de hebras por bloque predeterminado por el usuario y el mínimo número de bloques para procesar todo el array.\\
    \item El objetivo de cada bloque es calcular su \textit{scan} local.
\end{enumerate}
\begin{enumerate}[{2.}1]
    \item En primer lugar, cada uno de los \textit{warps} del bloque calcula su \textit{scan} local y lo almacena en un array. La operación utilizada para manejar los \textit{warps} es \textbf{shfl\_up\_sync}. Esta operación nos permite realizar una copia y desplazamiento hacia la derecha en el warp en función de una máscara, un valor y un desplazamiento. La máscara nos permite es un conjunto de 32 bits que nos permite indicar cuáles de las 32 hebras han de ser usadas. Para nuestra operación, la máscara está activa para todos las hebras. El valor indica qué valores vamos a utilizar para realizar la operación y ese valor será devuelto en caso de salirnos del rango del warp. Por último, el desplazamiento nos sirve para calcular cuántas posiciones nos hemos de desplazar hacia la derecha. Por ello, empezamos con un desplazamiento de 1 y vamos ampliando siguiendo las potencias binarias inferiores a 32. Para que los resultados ya calculados no sean modificados, nos aseguramos de trabajar sólo con las hebras dentro del warp con índice superior o igual al desplazamiento. Así, al tomar el desplazamiento 1, todos los elementos se suman a su anterior y el primero se mantiene constante. Al tomar el desplazamiento 2, sólo tenemos qué realizar la suma del que había dos posiciones antes, y así, sucesivamente, hasta llegar al último valor. Los últimos valores de cada \textit{scan} local serán guardados en un array en memoria compartida.
    \item Una vez las sumas locales de los warps han sido realizadas, hemos de añadir la suma acumulada obtenida en los \textit{warps} previos para obtener el resultado final del bloque. Para ello, se realiza un \textit{scan} de los \textit{warps} previos, obteniendo el array con las sumas acumuladas previas y estas son aplicadas a los elementos del array correspondientes.\\
\end{enumerate}
\begin{enumerate}[3]
    \item Una vez tenemos el \textit{scan} de cada bloque, hemos de realizar un procedimiento similar para llevar el \textit{scan} local del bloque a todo el array. Esto ha sido realizado, o bien mediante sumas atómicas en el mismo \textit{kernel}, que son más lentas que las operaciones normales, o bien combinando el trabajo con otros \textit{kernels} que tenían que ser lanzados de forma independiente.

\end{enumerate}



\subsection{Cálculo del criterio de Gini.}
Dado que sólo vamos a calcular el criterio de Gini para problemas de clasificación binaria hemos simplificado el mismo para ahorrarnos algunas operaciones a la hora de realizar el cálculo:

$$CRITERIO(A,v) = \frac{|i: A_i \leq v|}{N} \cdot GINI(|i: A_i \leq v|) + \frac{|i: A_i > v|}{N} \cdot GINI(|i: A_i > v|)$$

$$GINI(D) = 1 - \frac{T_D^2}{N_D^2} - \frac{F_D^2}{N_D^2}$$

Siendo $N_D$ el total de las muestras en el nodo $D$, $T_D$ el total de muestras pertenecientes a la clase positiva y $F_D$ el total de muestras pertenencientes a la clase negativa. Tenemos que:

$$F_D = N_D - T_D$$
Sustituyendo obtenemos que:

$$GINI(D) = \frac{N_D^2-T_D^2- (N_D - T_D)^2}{N_D^2} = \frac{N_D^2 - T_D^2 - (N_D^2+T_D^2 - 2 N_D T_D)}{N_D^2}$$
$$GINI(D) = \frac{-2T_D^2 + 2N_DT_D}{N_D^2} = 2 \frac{T_D(N_D-T_d)}{N_D^2}$$

$$CRITERIO = \frac{N_\leq}{N}\frac{2T_\leq(N_\leq - T_\leq)}{N_\leq^2} + \frac{N_>}{N}\frac{2T_>(N_> - T_>)}{N_>^2}$$
$$CRITERIO = \frac{2}{N}\Big(\frac{T_\leq(N_\leq - T_\leq)}{N_{\leq}} + \frac{T_>(N_> - T_>)}{N_>}\Big)$$

Puesto que además, no es de nuestro interés el valor específico sino obtener el valor óptimo, podemos ahorrarnos la multiplicación por $\frac{2}{N}$. Así pues, calculamos el criterio de la siguiente manera:

$$CRITERIO' = \Big(\frac{T_\leq(N_\leq - T_\leq)}{N_{\leq}} + \frac{T_>(N_> - T_>)}{N_>}\Big)$$

El valor de $CRITERIO'$ oscilará entre 0 y $\frac{N}{2}$ y buscaremos siempre obtener el mínimo valor para este criterio. Dicha búsqueda se realizará de manera similar a la realizada para el modelo anterior con una reducción para encontrar el índice mínimo en cada nodo.

\subsection{Reorganización de la listas de atributos.}
Para finalizar la evaluación de los nodos de un nivel podemos volver a aprovechar la operación de \textit{scan} para reorganizar el orden de los elementos de la lista de atributos sin necesidad de ejecutar ningún algoritmo de ordenación.\\

 Una vez se ha seleccionado la combinación de mejor lista de atributos para un nodo y su punto de corte, puesto que esta lista ya estaba ordenada, todos los elementos hasta el punto de corte pertenecen al nodo hijo izquierdo y los posteriores al nodo hijo derecho. Además, puesto que tenemos en la lista de atributos el campo ``ID Muestra'', podemos generar fácilmente un array de booleanos donde cada elemento indica si la muestra con el ID asociado a su posición.\\

 Si aplicamos la operación de \textit{scan} sobre este array auxiliar, la suma acumulada en cada posición nos indica el número de elementos que pertenecen al nodo hijo codificado con la etiqueta positiva. Por lo que, partiendo de que previamente estaban ordenados podemos hacer la subdivisión y mantener el orden teniendo en cuenta que:\\

 \begin{enumerate}
    \item Si el elemento en cuestión pertenece al nodo hijo izquierda (codificado como positivo en el array auxiliar), su nueva posición en la lista de atributos sería la suma acumulada obtenida en el array auxiliar menos uno (para que el primer índice sea el 0).
    \item En caso contrario, la nueva posición es el número total de elementos en el nodo hijo izquierda (para que ambos queden separados) más la diferencia de ese total de elementos del nodo hijo izquierda con la suma acumulada del array auxiliar (que nos indicaría cuántos elementos de la clase negativa llevamos hasta el momento) menos uno (porque la indexación empieza en 0).
 \end{enumerate}

\subsection{Representación del árbol.}
Durante el proceso de entrenamiento, la necesidad de lanzar múltiples \textit{kernels} al no disponer desde Python de paralelismo dinámico ha hecho que hayamos optado por almacenar tanto la estructura del árbol de salida como la que controla los nodos a evaluar en la CPU.\\

La estructura que controla los nodos a evaluar es una lista que contiene todos los nodos activos del nivel y, cada elemento de la lista, es una tupla que contiene el inicio, el fin del nodo y el índice correspondiente a ese nodo si recorremos el nivel de izquierda a derecha y no se hubiese podado ningún elemento. Puesto que es necesario que la CPU acceda a estos datos para controlar el flujo de ejecución de los \textit{kernels} no tiene sentido almacenar la misma en la GPU. \\

En el caso del árbol de salida, deberíamos de almacenar una estructura que nos permita almacenar los valores y atributos de un punto de corte y la clase de salida, si fuese un nodo terminal. Esta estructura podría ser almacenada por la GPU y podría ayudar a obtener mejores resultados, especialmente si disponemos de paralelismo dinámico. Sin embargo, tanto el momento en el que se poda un nodo como en el que se calcula el mejor punto de corte suponen la finalización de un \textit{kernel} y la devolución de datos a la CPU, por lo que añadir un dato más no supondría el cuello de botella si no la mera parada para realizar la transferencia. Además, en temas de escalabilidad, no almacenar la estructura del árbol en la GPU evita un gasto considerable de memoria RAM del dispositivo en un algoritmo con una gran complejidad espacial debido al uso de las listas de atributos en la GPU, que añaden dos campos extras para cada combinación de muestra y atributo que no sea la etiqueta de salida.\\

La representación utilizada para el árbol de salida es una lista de diccionarios de tuplas. En primer lugar, la lista tendrá tantos diccionarios como niveles de profundidad tenga el árbol, siendo el nivel 0 la raíz. El diccionario tendrá como clave de entrada un valor numérico, que se corresponderá a la posición que ocuparía el nodo si recorremos el nivel de izquierda a derecha y no se hubiese podado ningún elemento. De esta manera, si nos encontramos ante un nodo de decisión, sus hijos se encontrarían en el diccionario de la siguiente posición de la lista y sus índices de acceso serían el doble del índice de acceso de su padre o el doble del índice de acceso de su padre más 1, dependiendo de a cuál de los dos hijos queramos acceder. Por último, la descripción que se encuentra en en el diccionario para una clave de acceso es una tupla. Dicha tupla indica si el nodo es decisión o terminal. En el caso de ser un nodo de decisión, dispone de campos para el atributo y su valor de corte. En el caso de ser un nodo terminal, dispone de un campo que indica la clase con la que etiquetar la muestra. 

\subsection{Uso de Spark.}
Este modelo, al requerir la evaluación independiente de múltiples nodos, como podremos comprobar posteriormente, no escala bien con la generación de árboles profundos o completos. Es por eso que, a la hora de integrar \textit{Spark} en la solución, en vez de generar un único árbol, vamos a generar un \textit{random forest}, es decir, vamos a subdividir la muestras de entrenamiento aleatoriamente de tal manera que cada partición de \textit{Spark} entrenará un árbol y, una vez los árboles han sido entrenados, una muestra a clasificar será evaluada por todos los árboles. La clase de la muestra se corresponderá a la clase que ha sido seleccionada en el mayor número de árboles. \\

Esta solución nos permite, por un lado, reducir problemas de sobreajuste, y, por otro, conseguir precisiones competentes sin necesidad de generar árboles completos u otros sistemas de poda más complejos y evitar los problemas de sincronización y comunicación que generaría el entrenamiento de un único árbol como que todos los nodos de \textit{Spark} tendrían que tener una copia de las listas de atributos y, al terminar cada nivel de profundidad debería de plantearse una estrategia para reordenar las listas de atributos y volver a distribuir los cambios a todos los nodos.


\chapter{Desarrollo de pruebas y análisis de resultados.}
\section{Entorno de pruebas.}
Para el desarrollo de las pruebas, mi ordenador personal ha sido utilizado. Las especificaciones ténicas relevantes del mismo son:

\begin{itemize}
\item \textbf{Placa Base:} MSI B450M Bazooka
\item \textbf{Sistema Operativo:} Windows 10 64 bits
\item \textbf{CPU:} AMD Ryzen 5 2600X
\item \textbf{RAM:} Kingston HyperX Fury Black DDR4 2400 MHz PC4-19200 8GB CL15
\item \textbf{GPU}: \underline{Zotac GeForce GTX 1060 AMP! Edition}
\subitem \textbf{Núcleos CUDA}: 1280
\subitem \textbf{Frecuencia del procesador:} 1556 MHz (1771 MHz Boost)
\subitem \textbf{Frecuencia de la memoria:} 8 GHz
\subitem \textbf{Memoria}: 6 GB DDR5
\subitem \textbf{Bus de memoria:} 192-bit
\subitem \textbf{Compute Capability:} 6.1

\end{itemize}

\newpage

\section{Conjuntos de datos utilizados.}

Durante la fase de desarrollo del mapa autoorganizado hemos utilizado el conjunto de datos de las \textbf{caras de Olivetti}, creado por \textit{AT\&T Laboratories Cambridge} y descargada a través del paquete de Python \textit{scikit-learn} \cite{olivetti}. Dicho conjunto de imágenes consiste en 400 imágenes de 40 sujetos en escala de grises. Cada muestra son los valores de intensidad de cada píxel con un valor normalizado entre 0 y 1. Además, se proporciona una etiqueta que indica a qué sujeto pertenece cada imagen pero para los propósitos de nuestro modelo de aprendizaje no supervisado la misma no será utilizada. Las imágenes están en una versión cuadrada de 64x64 píxeles dándonos un total de 4096 valores de intensidad por muestra. \\

Durante la fase de desarrollo del árbol de decisión hemos utilizado dos conjuntos de datos de problemas de clasificación binaria: \textbf{Spambase} y \textbf{MAGIC Gamma Telescope}.\\

\textbf{Spambase} \cite{spambase} es un conjunto de 4601 muestras con 57 atributos. El objetivo en este conjunto de muestras es diferenciar correos no deseados (\textit{spam}) de correos deseados en función de los 56 atributos numéricos basados en el contenido del correo electrónico asociado a la muestra.\\

\textbf{Magic Gamma Telescope} \cite{magic04} es un conjunto de 19020 muestras con 11 atributos. Los datos de este conjunto fueron obtenidos en un experimento con un telescopio especial para observar rayos gamma de alta energía. El objetivo es diferenciar imágnes tomadas por el telescopio y preprocesadas de estas muestras generadas por rayos gamma (\textit{signal}) de las de rayos cósmicos en la capa superior de la atmósfera (\textit{background}). Los datos de este conjunto fueron generados a partir de simulaciones de Monte Carlo.\\

Para evaluar el rendimiento de ambos modelos para conjuntos de \textit{Big Data} hemos utilizado \textbf{SUSY} \cite{susy}. Este conjunto de datos contiene 5 millones de muestras con 18 atributos, que se generó a partir de un experimento de física en el que también se intenta diferencia un proceso que genera partículas supersimétricas (\textit{signal}) de otro proceso que no las genera (\textit{background}). En el caso del mapa autoorganizado la clase de salida es ignorada. De manera similar al anterior, los datos del conjunto fueron generadors a partir de simulaciones de Monte Carlo.


\section{Experimentos para evaluar el mapa autoorganizado.}
En el caso del mapa autoorganizado tanto la versión como para CPU como para GPU ejecutan el mismo algoritmo por lo que la métrica de interés durante las ejecuciones realizadas es el tiempo de ejecución. En primer lugar, durante la fase de desarrollo usamos el conjunto de las caras de \textit{Olivetti}, que nos permitió comprobar de manera empírica y gráfica que los resultados son correctos. En caso de funcionar, correctamente generamos un conjunto de imagenes con la misma dimensión del mapa que son o se parecen a algunas de las caras de los sujetos y donde las imágenes más parecidas se encuentran próximas las unas con las otras. \\

\textit{NOTA: Pendiente de ejecutar un ejemplo representantivo para un mapa de neuronas apropiado para un papel A4.}

Posteriormente, para evaluar la capacidad del algoritmo ante un conjunto de mayores dimensiones utilizamos SUSY. Para este experimento ignoramos las etiquetas de salida y utilizamos para un mapa de neuronas de 8 filas y 7 columnas con los parámetros de control $\tau$ a 10 y $\sigma_0$ a 4. El algoritmo lo ejecutamos durante 10 iteraciones y realizamos 5 repeticiones de cada experimento para tomar una medida de tiempo promedio más fiable. En este experimento no nos limitamos sólo a evaluar con el conjunto de muestras al completo (5 millones de muestras) sino que probamos desde medio millón de muestras hasta el conjunto completo aumentando en cada paso medio millón de muestras. Los resultados de este experimentos los podemos observar en las siguientes figuras.\\

\begin{figure}[ht]
\centering
\includegraphics[scale=0.8]{imagenes/tiempossusysom.png}
\caption{Tiempos de ejecución del mapaautoorganizado para SUSY.}
\label{img:somsusy}
\end{figure}

\begin{figure}[ht]
\centering
\includegraphics[scale=0.8]{imagenes/gananciassom.png}
\caption{Ganancias del mapaautoorganizado para SUSY.}
\label{img:gananciassomsusy}
\end{figure}

En este experimento podemos ver cómo evoluciona la ganancia de nuestra implementación que combina \textit{CUDA} y \textit{Spark} para evaluar un número de muestras elevado. En primer lugar, hemos de tener en cuenta que, al leer el archivo CSV, \textit{Spark} generó un \textit{RDD} con 72 particiones, es decir, 6 particiones para cada uno de los 12 núcleos (6 físicos y 6 lógicos) de nuestro AMD Ryzen 5 2600X. A la hora de evaluar con la GPU hemos mantenido el mismo número de particiones, por lo que en al evaluar 500000 en realidad estamos realizando 72 ejecuciones independientes del algoritmo y combinando todos los resultados obtenidos. Mientras que la generación de múltiples particiones facilita que los conflictos que se puedan generar para realizar las operaciones atómicas sean menores, resulta claro que conforme aumentamos el tamaño de la muestra obtenemos ganancias considerablemente mayores, como podemos observar en la figura \ref{img:gananciassomsusy}. Es por ello, que, a la hora de utilizar este algoritmo en la práctica hemos de concluir que hemos de exprimir al máximo las capacidades de la memoria del dispositivo CUDA para obtener las mejores ganancias. En nuestro caso, en el ejemplo más complejo que hemos evaluado obtenemos un tiempo de ejecución casi 8 veces más rápido en nuestra GTX 1060 que en los 12 cores de nuestro Ryzen.

\newpage
\section{Experimentos para evaluar el árbol de decisión.}
Para los experimentos relaciones con estos problemas de clasificación tenemos en cuenta dos métricas: el tiempo de ejecución y la precisión (porcentaje de predicciones correctas).\\

En primer lugar, durante el proceso de desarrollo, se empezó a evaluar la creación de un único árbol con profundidad limitada entre la implementación desarrollada y la implementación de Spark. Hemos de destacar, antes de comentar los resultados obtenidos, que, a diferencia del modelo anterior, el algoritmo utilizado para el cálculo en la CPU y el creado en CUDA no hacen lo mismo (el nuestro esta basado en una búsqueda exhaustiva de puntos de corte y el de Spark en una discretización por cuantiles y el uso de histogramas) aunque ambos generen un árbol de decisión. Los resultados obtenidos fueron generados mediante un proceso de validación cruzada \textit{leave-one out} con 10 particiones, es decir se generaron 10 particiones aleatorias de los datos y, en cada iteración, una es seleccionada para comprobar la precisión del modelo obtenido y el resto para entrenarlo, de tal manera que todas las particiones son utilizada para comprobar la calidad del modelo generado por el resto de ellas. Los resultados obtenidos de la validación cruzada son el promedio de estas iteraciones.
\newpage
\subsection{Tablas de resultados para la generación de un único árbol.}
\begin{table}[ht]
\begin{tabular}{@{}r|c|c|c|c|@{}}
\cmidrule(l){2-5}
\multicolumn{1}{c|}{\textbf{}}                      & \multicolumn{4}{c|}{\textit{\textbf{SPAMBASE (4601 muestras, 57 atributos)}}}                                                                                                                                         \\ \cmidrule(l){2-5} 
\multicolumn{1}{l|}{}                               & \multicolumn{2}{c|}{\textit{\textbf{CUDA}}}                                                               & \multicolumn{2}{c|}{\textit{\textbf{SPARK}}}                                                              \\ \midrule
\multicolumn{1}{|l|}{\textit{\textbf{Profundidad}}} & \textit{\textbf{\begin{tabular}[c]{@{}c@{}}Tiempo \\ (s)\end{tabular}}} & \textit{\textbf{\begin{tabular}[c]{@{}c@{}}Precisión \\ (\%)\end{tabular}}} & \textit{\textbf{\begin{tabular}[c]{@{}c@{}}Tiempo \\ (s)\end{tabular}}} & \textit{\textbf{\begin{tabular}[c]{@{}c@{}}Precisión \\ (\%)\end{tabular}}} \\ \midrule
\multicolumn{1}{|r|}{\textit{\textbf{4}}}           & \textbf{2,58}                                     & 88,13                                                 & 6,47                                              & \textbf{88,64}                                        \\ \midrule
\multicolumn{1}{|r|}{\textit{\textbf{5}}}           & \textbf{2,92}                                     & 85,7                                                  & 6,48                                              & \textbf{90,77}                                        \\ \midrule
\multicolumn{1}{|r|}{\textit{\textbf{6}}}           & \textbf{4,44}                                     & 86,2                                                  & 6,6                                               & \textbf{91,14}                                        \\ \midrule
\multicolumn{1}{|r|}{\textit{\textbf{7}}}           & \textbf{7,32}                                     & 89,72                                                 & 6,75                                              & \textbf{91,9}                                         \\ \midrule
\multicolumn{1}{|r|}{\textit{\textbf{8}}}           & 10,81                                             & 90,15                                                 & \textbf{6,84}                                     & \textbf{91,95}                                        \\ \midrule
\multicolumn{1}{|r|}{\textit{\textbf{9}}}           & 12,15                                             & 85,8                                                  & \textbf{6,9}                                      & \textbf{92,29}                                        \\ \bottomrule
\end{tabular}
\caption{Validación cruzada para árbol de decisión en spambase.}
\label{tab:spamtree}
\end{table}

\begin{table}[ht]
\begin{tabular}{@{}r|c|c|c|c|@{}}
\cmidrule(l){2-5}
\multicolumn{1}{c|}{\textbf{}}                      & \multicolumn{4}{c|}{\textit{\textbf{MAGIC (19020 muestras, 11 atributos)}}}                                                                                                                                                                                                                                   \\ \cmidrule(l){2-5} 
\multicolumn{1}{l|}{}                               & \multicolumn{2}{c|}{\textit{\textbf{CUDA}}}                                                                                                           & \multicolumn{2}{c|}{\textit{\textbf{SPARK}}}                                                                                                          \\ \midrule
\multicolumn{1}{|l|}{\textit{\textbf{Profundidad}}} & \textit{\textbf{\begin{tabular}[c]{@{}c@{}}Tiempo \\ (s)\end{tabular}}} & \textit{\textbf{\begin{tabular}[c]{@{}c@{}}Precisión \\ (\%)\end{tabular}}} & \textit{\textbf{\begin{tabular}[c]{@{}c@{}}Tiempo \\ (s)\end{tabular}}} & \textit{\textbf{\begin{tabular}[c]{@{}c@{}}Precisión \\ (\%)\end{tabular}}} \\ \midrule
\multicolumn{1}{|r|}{\textit{\textbf{4}}}           & \textbf{0,36}                                                           & 79,37                                                                       & 6,42                                                                    & \textbf{81,41}                                                              \\ \midrule
\multicolumn{1}{|r|}{\textit{\textbf{5}}}           & \textbf{0,65}                                                           & 80,99                                                                       & 6,48                                                                    & \textbf{81,71}                                                              \\ \midrule
\multicolumn{1}{|r|}{\textit{\textbf{6}}}           & \textbf{1,17}                                                           & 81,77                                                                       & 6,58                                                                    & \textbf{83,59}                                                              \\ \midrule
\multicolumn{1}{|r|}{\textit{\textbf{7}}}           & \textbf{2,09}                                                           & 83,87                                                                       & 6,71                                                                    & \textbf{84,23}                                                              \\ \midrule
\multicolumn{1}{|r|}{\textit{\textbf{8}}}           & \textbf{3,52}                                                           & 84,1                                                                        & 6,81                                                                    & \textbf{84,55}                                                              \\ \midrule
\multicolumn{1}{|r|}{\textit{\textbf{9}}}           & \textbf{5,77}                                                           & 84,2                                                                        & 7,05                                                                    & \textbf{84,73}                                                              \\ \bottomrule
\end{tabular}
\caption{Validación cruzada para árbol de decisión en MAGIC.}
\label{tab:magictree}
\end{table}
\newpage

\subsection{Análisis de los resultados para la generación de un único árbol.}
\begin{figure}[ht]
\centering
\includegraphics[scale=1.0]{imagenes/spambase_times.png}
\caption{Tiempo de entrenamiento y ganacia según profundidad en Spambase.}
\label{img:spambasetimes}
\end{figure}
En la figura \ref{img:spambasetimes} podemos observar cómo evolucionan los tiempos de ejecución del modelo para CPU (en azul) y CUDA (en verde) según vamos aumentando la profundidad de los árboles generados empezando en cuarto nivel de profundidad y terminando en el noveno. Podemos concluir con facilidad que el rendimiento de la implementación en CUDA empeora considerablemente conforme aumentamos los niveles de profundidad, incluso llegando a ser más lento que la versión en CPU de \textit{Spark}, factor que resulta razonable pues, conforme vamos aumentando dicho nivel más kernels han de ser lanzados y el número de muestras en cada uno va a ser inferior pudiendo llegar a ser incapaz de aprovechar la capacidad de procesamiento que tienen los núcleos de CUDA por ejemplo sin el nodo no hay suficientes elementos para utilizar todas las hebras de un bloque o siquiera las de un \textit{warp}, así como la reorganización de los datos en la memoria que sigue siendo una operación bastante costosa, aunque no haga falta aplicar de nuevo un algoritmo de ordenación, que ocurre en cada nivel. \\

La generación de un árbol de decisión para Spambase expone los dos problemas principales cuando utilizamos este algoritmo: tener un número de muestras relativamente reducido y una gran cantidad de listas de atributos. Si tenemos un número de muestras más o menos reducido, en este caso tenemos 4600 muestras, el nivel de profundidad én el que la utilización de la GPU de manera efectiva decae considerablemente se alcanza antes. Por otro lado, el elevado número de atributos (56 más la etiqueta de salida) implica la generación de 56 listas de atributos que generan tanto un \textit{overhead} en la complejidad espacial del modelo en la GPU como en la cantidad de transferencias de datos que se han de realizar en la memoria global del dispositivo CUDA. \\

\begin{figure}[ht]
\centering
\includegraphics[scale=1.0]{imagenes/magic_times.png}
\caption{Tiempo de entrenamiento y ganacia según profundidad en MAGIC.}
\label{img:magictimes}
\end{figure}

En la figura \ref{img:magictimes} realizamos el mismo análisis para el conjunto de datos \textit{MAGIC Gamma Telescope}. En este caso nos encontramos, en cierta manera, con la situación contraria a la que observábamos en Spambase, el número de muestras considerablemente más elevado (19020) y el número de atributos mucho más reducido (10 más la etiqueta de salida) ponen de manifiesto la velocidad de la ejecución del algoritmo cuando se dan las condiciones ideales para su uso. Cabe destacar que para profundidades muy reducidas, como ocurre en el nivel de profundidad 4, obtenemos una gran ganancia, siendo 17 veces más rápida nuestra implementación que la de \textit{Spark} y manteniendo ganancias considerables para profundidades 6 y 7 (5,62 y 3,21, respectivamente). \\

Puesto que, como comentábamos previamente, ambas implementaciones no hacen exactamente lo mismo es importante observar también las diferencias existentes en términos de precisión para ambos modelos.\\

\begin{figure}[ht]
\centering
\includegraphics[scale=1.0]{imagenes/spambase_prec.png}
\caption{Precisión según profundidad en SPAMBASE.}
\label{img:spambaseprec}
\end{figure}

La precisión de nuestra implementación según la profundidad (figura \ref{img:spambaseprec}) es considerablemente inferior en la mayoría de niveles de profundidad para el entrenamiento de \textit{Spambase}. La posibilidad de que muestras ruidosas afecten a la calidad de los resultados obtenidos en nuestro modelo es mucho mayor que en el de \textit{Spark} pues no hemos utilizado ninguna técnica de poda más avanzada que limitar la profundidad del árbol generado. Dependiendo del nivel de profundidad, nuestra implementación llega a un rango competitivo de precisión con \textit{Spark} o queda considerablemente por detrás, siendo el caso más claro la profundidad 9 donde \textit{Spark} obtiene una ventaja en precisión del 6,49 \%.\\

\begin{figure}[ht]
\centering
\includegraphics[scale=1.0]{imagenes/magic_prec.png}
\caption{Precisión según profundidad en MAGIC.}
\label{img:magicprec}
\end{figure}

Por otro lado, en la figura \ref{img:magicprec}, podemos observar que no sólo podemos tener problemas por la falta de técnicas de poda avanzadas sino que puesto que ambos algoritmos generan un árbol de decisión de maneras considerablemente distintas habrá situaciones en las que las diferencias entre uno y otro sean mayores y otras en las que sean ambas muy competentes, como en este caso que según el nivel de profundidad la diferencia de precisión varía entre el 0,36 \% y el 2,04 \%.\\

Una vez vistos los puntos fuertes y débiles de nuestra implementación a la hora de probar con nuestro conjunto de \textit{Big Data}, \textit{SUSY}, decidimos en vez de generar un único árbol generar un \textit{random forest} con nuestros árboles de profundidad limitada. En concreto, para el entrenamiento tenemos 4 millones y medio de muestras de 18 atributos más la etiqueta de clasificación. Para nuestro experimento sobre este conjunto hemos generado un \textit{random forest} de 225 árboles utilizando cada uno de ellos 20000 de la muestras presentes en el conjunto y utilizando todos los atributos para entrenar cada árbol. Hemos realizado la misma validación cruzada y realizado el experimento para el sexto y el séptimo nivel de profundidad.

\newpage
\subsection{Tabla de resultados del random forest.}

\begin{table}[ht]
\begin{tabular}{@{}l|cc|cc|@{}}
                             & \multicolumn{2}{c|}{\textit{\textbf{CUDA}}}                                                              & \multicolumn{2}{c|}{\textit{\textbf{SPARK}}}                                                             \\ \midrule
\textit{\textbf{Repetición}} & \multicolumn{1}{l}{\textit{\textbf{Tiempo (s)}}} & \multicolumn{1}{l|}{\textit{\textbf{Precisión (\%)}}} & \multicolumn{1}{l}{\textit{\textbf{Tiempo (s)}}} & \multicolumn{1}{l|}{\textit{\textbf{Precisión (\%)}}} \\ \midrule
\textit{\textbf{1}}          & \textbf{560,48}                                  & 78,12                                                 & 1201,04                                          & \textbf{85,3}                                         \\
\textit{\textbf{2}}          & \textbf{536,01}                                  & 78,14                                                 & 1273,04                                          & \textbf{85,3}                                         \\
\textit{\textbf{3}}          & \textbf{519,96}                                  & 78,24                                                 & 1236,47                                          & \textbf{85,38}                                        \\
\textit{\textbf{4}}          & \textbf{519,47}                                  & 78,15                                                 & 1238,79                                          & \textbf{85,33}                                        \\
\textit{\textbf{5}}          & \textbf{530,59}                                  & 78,13                                                 & 1334,8                                           & \textbf{85,33}                                        \\
\textit{\textbf{6}}          & \textbf{543,35}                                  & 78,11                                                 & 1324,34                                          & \textbf{85,24}                                        \\
\textit{\textbf{7}}          & \textbf{545,54}                                  & 78,15                                                 & 1187,43                                          & \textbf{85,27}                                        \\
\textit{\textbf{8}}          & \textbf{530,74}                                  & 78,1                                                  & 1247,09                                          & \textbf{85,29}                                        \\
\textit{\textbf{9}}          & \textbf{558,46}                                  & 78,18                                                 & 1251                                             & \textbf{85,34}                                        \\
\textit{\textbf{10}}         & \textbf{521,29}                                  & 78,1                                                  & 1268,34                                          & \textbf{85,28}                                        \\ \midrule
\textit{\textbf{MEDIA}}      & {\ul \textit{\textbf{536,59}}}                   & \textit{\textbf{78,14}}                               & \textit{\textbf{1256,23}}                        & {\ul \textit{\textbf{85,31}}}                         \\ \bottomrule
\end{tabular}
\caption{Resultados de validación cruzada en SUSY para profundidad 6.}
\label{tab:susyprof6}
\end{table}

\begin{table}[ht]
\begin{tabular}{@{}l|cc|cc|@{}}
                             & \multicolumn{2}{c|}{\textit{\textbf{CUDA}}}                                                              & \multicolumn{2}{c|}{\textit{\textbf{SPARK}}}                                                             \\ \midrule
\textit{\textbf{Repetición}} & \multicolumn{1}{l}{\textit{\textbf{Tiempo (s)}}} & \multicolumn{1}{l|}{\textit{\textbf{Precisión (\%)}}} & \multicolumn{1}{l}{\textit{\textbf{Tiempo (s)}}} & \multicolumn{1}{l|}{\textit{\textbf{Precisión (\%)}}} \\ \midrule
\textit{\textbf{1}}          & \textbf{583,13}                                  & 78,84                                                 & 1563,44                                          & \textbf{85,78}                                        \\
\textit{\textbf{2}}          & \textbf{559,27}                                  & 78,83                                                 & 1692,27                                          & \textbf{85,8}                                         \\
\textit{\textbf{3}}          & \textbf{584,07}                                  & 78,97                                                 & 1768,84                                          & \textbf{85,88}                                        \\
\textit{\textbf{4}}          & \textbf{548,74}                                  & 78,86                                                 & 1752,11                                          & \textbf{85,82}                                        \\
\textit{\textbf{5}}          & \textbf{561,02}                                  & 78,78                                                 & 1681,44                                          & \textbf{85,82}                                        \\
\textit{\textbf{6}}          & \textbf{574,28}                                  & 78,76                                                 & 1734,05                                          & \textbf{85,72}                                        \\
\textit{\textbf{7}}          & \textbf{572,08}                                  & 78,84                                                 & 1790,56                                          & \textbf{85,77}                                        \\
\textit{\textbf{8}}          & \textbf{580,59}                                  & 78,81                                                 & 1769,53                                          & \textbf{85,76}                                        \\
\textit{\textbf{9}}          & \textbf{576,32}                                  & 78,81                                                 & 1816,56                                          & \textbf{85,82}                                        \\
\textit{\textbf{10}}         & \textbf{592,13}                                  & 78,77                                                 & 1175,02                                          & \textbf{85,76}                                        \\ \midrule
\textit{\textbf{MEDIA}}      & {\ul \textit{\textbf{573,16}}}                   & \textit{\textbf{78,83}}                               & \textit{\textbf{1674,38}}                        & {\ul \textit{\textbf{85,79}}}                         \\ \bottomrule
\end{tabular}
\caption{Resultados de validación cruzada en SUSY para profundidad 7.}
\label{tab:susyprof7}
\end{table}

\newpage
\subsection{Análisis de los resultados del random forest.}
A la hora de realizar este experimento podemos observar cómo los tiempos de ejecución de cada iteración se han disparado para afrontar el problema de mayores dimensiones. La generación de múltiples árboles pequeños nos ha llevado a tardar un promedio de 536,59 segundos para profundidad 6 y 573,16 segundos, es decir un periodo de entre 9 y 10 minutos utilizando la GPU. En el caso de la versión de \textit{Spark} para CPU, en profundidad 6 hemos tardado 1256,23 segundos (unos 20 minutos) y 1674,38 segundos para profundiad 7 (casi 28 minutos), dando lugar a una ganancia promedio de 2,34 para profundidad 6 y de 2,92 para profundidad 7. Mientras que en términos de velocidad de entrenamiento nuestro algoritmo ha sido considerablemente superior incluso incrementando la ganancia al pasar de un nivel de profundidad a otro, hemos de tener en cuenta también la diferencia existente en términos de precisión, que deja nuestra implementación un 7,16 \% de precisión peor para profundidad 6 y un 6,97 \% peor para profundidad 7.\\

\begin{figure}[ht]
\centering
\includegraphics[scale=1.0]{imagenes/boxplot_rf.png}
\caption{Diagrama de cajas y bigotes para el tiempo de entrenamiento del random forest para SUSY.}
\label{img:boxplot}
\end{figure}

De todos los experimentos realizados podemos concluir que mientras que, efectivamente nuestro modelo es más rápido siempre y cuando no generemos árboles demasiado profundos o tengamos una cantidad de atributos muy elevada. En este trabajo, nos hemos centrado en conseguir esa ganancia durante el proceso de entrenamiento. No obstante, pueden ser líneas de trabajo futuras sobre este modelo un análisis específico de los parámetros de profundidad y número de árboles óptimos, la utilización de otros criterios además del utilizado, o el uso del tiempo extra obtenido para mejorar los resultados ya sea con algún preprocesamiento de datos o el uso de criterios de poda más avanzados, entre otros.
%!TEX root = ../proyecto.tex
\chapter{Conclusiones y trabajos futuros.}
En este trabajo, hemos desarrollado dos algoritmos de \textit{Soft Computing}: el mapa auto-organizado de Kohonen y un árbol de decisión, realizado una adaptación de los mismos para dispositivos \textit{CUDA}, y evaluando los resultados obtenidos con conjuntos de datos masivos mientras utilizamos el \textit{framework Spark}, que nos ha llevado a convertir nuestro modelo que utiliza un único árbol de decisión a un \textit{random forest}.\\

En el caso del mapa auto-organizado de Kohonen, la combinación de GPU y \textit{Spark} proporciona resultados significativamente mejores que la versión para CPU, llegando a ser 27 veces más rápida la primera que la segunda en el mejor de los casos probados. Sin embargo, en el caso del \textit{random forest}, la implementación utilizada, aunque proporciona ligeras mejoras en velocidad si el número de datos a procesar es lo suficientemente grande, no llega a las proporciones del primer caso. Esto se ha debido a que, en el primer caso, hemos afrontado un problema que podía ser resuelto de forma masivamente paralela mientras que, en el segundo, múltiples factores como, por ejemplo, la necesidad de procesar múltiples nodos de forma independiente, han limitado las posibilidades del dispositivo \textit{CUDA}. \\

Por otro lado, el uso de \textit{Python} y \textit{Spark} ha resultado clave para poder desarrollar este proyecto de forma eficiente. \textit{Spark} nos ha proporcionado herramientas para trabajar con bases de datos masivas, integrarlo con \textit{CUDA} mediante Python y permitirnos realizar una implementación que podríamos llevar a un clúster con múltiples GPUs.\\

Las pruebas han sido realizadas sobre una única máquina y, aunque, en ambas pruebas el uso de la GPU con bases de datos masivas ha resultado favorable, no podemos concluir que el uso de \textit{CUDA} sea favorable siempre, aunque si podemos decir con certeza que, si el problema es masivamente paralelizable y se usa una base de datos lo suficientemente grande, el uso de la GPU será significativamente superior a los resultados obtenidos por la CPU.\\

Para terminar, vamos a comentar brevemente algunas vías de desarrollo por las que se podría ampliar el trabajo realizado: \\
\begin{itemize}
	\item Añadir nuevos algoritmos de \textit{Soft Computing} o más bases de datos sobre las que realizar pruebas.\\
	\item Realizar una comparativa de los resultados obtenidos en un rango de sistemas con especificaciones técnicas diferentes al utilizado.\\
	\item Realizar un análisis más profundo sobre las implicaciones de modificar ciertos parámetros utilizados en los algoritmos, como el número de particiones de \textit{Spark}, el tamaño del mapa de Kohonen o la profundidad de los árboles.\\
	\item Realizar una comparativa de los resultados obtenidos entre clúster de GPU y clúster de CPU.\\
\end{itemize} 

%\input{capitulos/ESTADOARTE}
%\chapter{Mapas autoorganizados.}
\section{Definición.}
Los mapas autoorganizados \textit{Self Organizing Map}, también llamados mapa autoorganizados de características \textit{Self Organizing Feature Map} o redes neuronales de Kohonen son un modelo neuronal de aprendizaje competitivo y no supervisado propuesto por \textit{Teuvo Kohonen} a principios de la década de 1980.\\

Al ser un modelo de aprendizaje no supervisado, se encarga de descrubir patrones iguales en las muestras y agruparlos basandose en la información de las características. 

\section{Arquitectura de una mapa autoorganizado.}
La arquitectura de este tipo de redes neuronales está formada por dos capas totalmente interconectadas. Una capa de entrada y una capa de salida (también llamada capa de Kohonen, capa competitiva o capa de aprendizaje). \\

La capa de entrada tiene el mismo número de nodos que el número de características que tiene cada muestra de entrada.\\

La capa de salida está compuesta por una estructura con $X$ neuronas, habitaulmente colocadas en una matriz bidimensional, aunque no tiene por qué ser así, donde cada una de estas neuronas tiene asociada un vector de pesos $W$ que une cada uno de los nodos de entrada con una neurona de la capa de salida.
\begin{figure}
\centering
\includegraphics[width=0.8\textwidth]{imagenes/arquitectura_som.png}
\caption{Esquema de una red neuronal de Kohonen.}
\end{figure}
\section{Proceso de entrenamiento de un mapa autoorganizado.}
\textit{En la explicación de este algoritmo se asume que la capa de salida es una matriz bidimensional, como se ha comentado antes, esto no tiene que ocurrir aunque es lo más común.}\\

En primer lugar, se \textbf{inicializan} \textbf{los pesos} asociados a la capa de salida. Lo más habitual, es tomar dichos pesos de una distribución aleatoria.
Para un correcto funcionamiento dichos pesos deben estar normalizados entre 0 y 1. En nuestro caso, hemos tomado los pesos de una distribución aleatoria uniforme en el intervalo $[0, 1)$.\\

El proceso de entrenamiento que se presenta a continuación \textbf{se repite hasta que se alcanza el número de iteraciones máximo}, determinado por el parámetro $\lambda$. La variable $t$ representa cada una de las iteraciones.\\

Después, para cada una de las muestras, $X$, sacadas de la distribución de muestras, de forma aleatoria, se realizan los siguientes pasos: \\\\
1 - Se \textbf{calcula la distancia euclídea} entre la muestra $X$ y cada una de las neuronas de la capa de salida.\\
$$distancia(X) = || W - X ||$$\\
2 - Se \textbf{busca la neurona que ha obtenido una menor distancia}. Esta neurona es considerada la neurona ganadora o BMU \textit{(Best Matching Unit)}.\\
$$BMU_X = argmin_{W_{i, j}} \; distancia(X) = argmin_{W_{i, j}} || W - X ||$$\\
3 - Se realiza un \textbf{proceso de actualización de las matrices de pesos} en base a lo obtenido anteriormente, según la siguiente fórmula.\\
$$ W_{i, j}^{(t+1)} = W_{i, j}^{(t)} + \Delta {W_{i,j}} $$\\

La actualización depende tanto de la distancia de la muestra al vector de pesos como de otros dos parámetros: la tase de aprendizaje y una función de vecindario.\\

$$\Delta W_{i,j} = \eta(t)\delta_f(i,j)(X-W_{i,j})$$\\

La función $\delta_f(i,j)$ es la función de vecindario y, en nuestra se calcula conforme a la siguiente ecuación:\\

$$\delta_f(i,j) = e ^{-\frac{||BMU_X-(i,j)||^2}{2\sigma(t)^2}}. $$\\

Al tratar con una potencia con exponente negativo, un mayor valor absoluto de dicho exponente nos proporciona una valor de $\delta_f(i,j)$ menor. Por eso en el numerador se tiene en cuenta la distancia que hay entre la mejor neurona y la neurona actual. En el denominador se utiliza un parámetro de control $\sigma$que nos permite controlar la distancia que estamos considerando.\\\\

Normalmente, este parámetro, durante un primer número de iteraciones previamente proporcionado, de las consideradas en el proceso de entrenamiento, es inicializado a un valor alto $\sigma_0$ que decrece de manera exponencial conforme a otro parámetro de control $\tau$.


Una vez ha finalizado esa primera fase (han pasado $z$ iteraciones) se van refinando los resultados con un valor fijo mucho más bajo $\sigma_f$.\\


$$\sigma(t) = \left\{
\begin{array}{ll}
\sigma_0e^{-\frac{t}{\tau}} & si \;\;t < z\\
\sigma_f & si  \;\; t\geq z
\end{array}
\right.
$$\\

Para la tasa de aprendizaje se sigue una aproximación similar, la tasa de aprendizaje durante la primera fase está inicializada a un valor $\eta_0$ decreciendo conforme a una gaussiana y, una vez pasado un número de iteraciones, es fijado a un valor $\eta_f$. \\
$$\sigma(t) = \left\{
\begin{array}{ll}
\eta_0e^{-\frac{t}{\tau}} & si \;\;t < z\\
\eta_f & si  \;\; t\geq z
\end{array}
\right.$$\\

Así pues, este algoritmo acerca los pesos del vecindario de la BMU hacia la nueva muestra introducida para parecerse más a la misma. Esto lo hace teniendo en cuenta un vecindario alrededor de la BMU que decrece exponencialmente conforme pasa un número de iteraciones hasta quedarse fijo y una tasa de aprendizaje que también decrece exponencialmente hasta permanecer constante. \\Esto permite una primera fase de entrenamiento, con cambios más bruscos en la que se adaptan los valores completamente aleatorios para encontrar agrupamientos razonables. Conforme avanza dicha fase esos valores van decreciendo, hasta que quedan fijados permitiendo a la red neuronal refinar los agrupamientos obtenidos hasta ese momento.



\section{Implementación en CUDA.}
Antes de realizar la primera implementación en CUDA, analizamos las posibilidades de paralelización que nos aporta el algoritmo así como los accesos a memoria requeridos para utilizar las herramientas que cuda nos proporciona de la forma más eficiente posible.

\subsection{Grado de paralelismo.}
El algoritmo muestra una posibilidad de implementarlo de forma paralela innata. Cada vez que una muestra es tomada de la entrada todas las neuronas realizan las mismas operaciones con esa entrada. Por ello, la componente principal que va a determinar el grado de paralelismo que podríamos alcanzar como máximo viene dado por el número de neuronas que hay en la capa de salida. \\

Otra opción que sería interesante sería que el paralelismo dependiera del número de muestras procesadas, ya que, es mucho más frecuente a la hora de utilizar este tipo de modelos, especialmente al aplicarlos con el fin de realizar \textit{clustering} (agrupamiento de datos) o reducción de dimensionalidad que el número de muestras sea considerablemente superior al número de neuronas (grupos) que se desean obtener en la capa de salida. \\

Sin embargo, la dependencia de datos existente para actualizar la matriz de pesos en cada iteración genera una barrera de comunicación que evita que podamos plantear una solución eficiente del algoritmo estudiando basándonos en las muestras.

\subsection{Uso de memoria.}
En esta sección vamos a analizar las operaciones relacionadas con la memoria para poder tomar decisiones óptimas con respecto a ellas. \\

\underline{Transferencia de elementos del host al dispositivo.}\\
Cada vez que se invoca uno de los kernels utilizados para la resolución del algoritmo acabamos de obtener una nueva muestra dentro de una iteración. Para poder aprovechar la gran capacidad de paralelismo de la GPU es necesario que dichos elementos se encuentren la GPU, por ello, en esta implementación vamos a mover todo el conjunto de muestras al dispositivo antes de empezar a calcular el algoritmo, evitando así que se interrumpa la ejecución constante del dispositivo para recibir nuevas entradas a través del PCI Express 3.0.

Los parámetros de control, se mantendrán todos en la CPU y será necesario enviarlos para el kernel que se encargue de actualizar la matriz de pesos.\\

\underline{Uso de memoria compartida.}\\
El uso de la memoria compartida en una GPU es fundamental para poder obtener los mejores resultados. Esta memoria compartida es mucho más rápida que la memoria global del sistema ya que, en vez de hacer uso de la DRAM del dispositivo, los elementos de esta memoria se alojan en la caché L1 de la tarjeta permitiendo un acceso tanto para escritura como para lectura muy superior pero limitando a 48 KB el máximo de elementos que se pueden alojar por bloque de hebras en la misma.

En el algoritmo propuesto encontramos dos operaciones en las que podemos hacer uso de la memoria compartida.

\begin{itemize}
	\item A la hora de calcular la distancia euclídea, podemos guardar la muestra que estamos evaluando en memoria compartida, asegurándonos de que esta permanezca en la caché.
	\item A la hora de buscar el índice de la menor distancia, que se va a realizar mediante una técnica de reducción y que, por tanto, conlleva el uso de la memoria compartida para obtener buenos resultados.
\end{itemize}

\underline{Uso de la memoria del dispositivo.}\\
Todos los elementos generados en el dispositivo, se mantendrán en el mismo sin necesidad de ser transladados al host, salvo la matriz de pesos solución que una vez finalizada la ejecución del algoritmo se enviará al host. \\

\subsection{Kernels a desarrollar.}
Para obtener los mejores resultados hemos de intentar lanzar el menor número posible de kernels para evitar así los pequeños overheads que se dan para lanzar los mismos. Sin embargo, la naturaleza del algoritmo, que repite el proceso constantemente dificulta esta labor. \\

Antes de iniciar el bucle del algoritmo lanzaremos el primer kernel.\\

\underline{I - Inicialización aleatoria de pesos \textit{(cuda\_init\_weights)}.}\\

En el primer kernel vamos a inicializar aleatoriamente el array de pesos con valores sacados de una distribución uniforme de números reales en coma flotante entre 0 y 1. La generación de números aleatorios se realiza utilizando funciones que nos proporciona la librería Numba, cuya implementación se basa en el método de Box-Muller. \\\\

Este array de una única dimensión representa una estructura tridimensional basado en las filas de la matriz, las columnas de la matriz y la dimensión de los vectores de pesos.\\\\\\

En cada iteración del algoritmo se ejecuta lo siguiente:\\

\underline{II - Selección aleatoria de las muestras a usar \textit{(en CPU)}}\\
Puesto que el tamaño de muestras que se toma por iteración no tiende a ser muy grande, hemos optado por realizar esta opción en la CPU.\\

\underline{III - Cálculo de la distancia euclídea. \textit{(cuda\_euclidean\_distance)}} \\
Este kernel recibe una muestra y calcula la distancia euclídea de esa muestra con todos los elementos de la matriz de pesos. Para hacerlo de una manera más eficiente primero se introduce la muestra en memoria compartida y luego se calcula la distancia.

El array en el que se guardan las distancias resultantes permanece siempre en la memoria del dispositivo y es reutilizado de una iteración a otra. \\

\underline{IV - Encontrar menor distancia \textit{(amin() cuBLAS.)}} \\
En principio para solucionar este problema habíamos planteado una reducción. La reducción es una técnica común utilizada para computar eficientemente en una GPU una operación binaria que cumple la propiedad asociativa sobre todos los elementos de un array. El ejemplo más habitual de ello, es la suma. Puesto que la operación binaria mínimo cumple ese prerrequisito hemos planteado y desarrollado la solución con una reducción en la que tenemos en cuenta la dupla que conforman el valor y su índice en el array.

Este ha sido el objeto de más dificultad de esta implementación y, aunque hemos conseguido una implementación considerablemente eficiente especialmente ante una implementación secuencial, hemos considerarlo sustituirlo por el método amin() de la librería altamente optimizada de cuBLAS que calculaba los resultados entre 2 y 4 veces más rápido (dependiendo del tamaño de la entrada), que nuestra implementación.\\

\underline{ V - Actualización de la matriz de pesos \textit{(cuda\_bmu\_update)}} \\
Este kernel recibe los parámetros de control y aplica la actualización de los pesos conforme a lo descrito en la fase de actualización del algoritmo explicado en la sección anterior (3.3).

%\chapter{Árboles de decisión.}

Un árbol de decisión es un modelo de aprendizaje automático no lineal que permite resolver los problemas de manera jerárquica. Una vez el árbol es construido el árbol consta de dos tipos de nodos: los nodos de decisión y los nodos terminales. Los \textbf{nodos de decisión}, son reglas que nos idican cómo hemos de navegar el árbol para clasificar una muestra. Lo habitual es que dicho nodo esté asociado a una carácterística del problema y un valor de tal manera que si la condición se cumple, tomamos un camino y, en caso contrario, tomamos el otro. Una vez alcanzamos un \textbf{nodo terminal} (correspondientes a las hojas de los árboles), conocemos la categoría a la que corresponde una muestra.\\\\

La mayoría de los algoritmos secuenciales utilizan metodologías de tipo voraz \textit{(greedy)} para la generación de manera recursiva de la estructura del árbol. Dentro de esta categoría encontramos los algoritmos más conocidos como ID3, C4.5 y CART.

%\input{capitulos/01_Introduccion}
%
%\input{capitulos/02_EspecificacionRequisitos}
%
%\input{capitulos/03_Planificacion}
%
%\input{capitulos/04_Analisis}
%
%\input{capitulos/05_Diseno}
%
%\input{capitulos/06_Implementacion}
%
%\input{capitulos/07_Pruebas}
%
%\input{capitulos/08_Conclusiones}
%
%%\chapter{Conclusiones y Trabajos Futuros}
%
%
%%\nocite{*}
%\bibliographystyle{miunsrturl} 
\bibliographystyle{IEEEtran} 
\bibliography{bibliografia/bibliografia}\addcontentsline{toc}{chapter}{Bibliografía}
%

%\input{apendices/manual_usuario/manual_usuario}
%\chapter{ESPECIFICACIONES DEL SISTEMA}
Aquí se especifican las características del sistema utilizado para las pruebas.

\begin{itemize}
\item \textbf{Placa Base:} MSI B450M Bazooka
\item \textbf{Sistema Operativo:} Windows 10 64 bits
\item \textbf{CPU:} AMD Ryzen 5 2600X
\item \textbf{RAM:} Kingston HyperX Fury Black DDR4 2400 MHz PC4-19200 8GB CL15
\item \textbf{GPU}: \underline{Zotac GeForce GTX 1060 AMP! Edition}
\subitem \textbf{Núcleos CUDA}: 1280
\subitem \textbf{Frecuencia del procesador:} 1556 MHz (1771 MHz Boost)
\subitem \textbf{Frecuencia de la memoria:} 8 GHz
\subitem \textbf{Memoria}: 6 GB DDR5
\subitem \textbf{Bus de memoria:} 192-bit
\subitem \textbf{Compute Capability:} 6.1


\end{itemize}
%\input{glosario/entradas_glosario}
% \addcontentsline{toc}{chapter}{Glosario}
% \printglossary

\thispagestyle{empty}

\end{document}


 