\section{Breve introducción a CUDA.}
Como comentábamos al principio, CUDA \textit{(Computer Unified Device Arquitecture)} \cite{cuda} es una tecnología propietaria desarrollada por \textit{NVIDIA} y lanzada en junio de 2007, que nos proporciona de un lenguaje de programación general destinado a ser ejecutado en las tarjetas gráficas de la compañia. Para los propósitos de este trabajo y, habitualmente, a la hora de trabajar con CUDA denominaremos como \textbf{\textit{host}} a la CPU que se comunica con la tarjeta gráfica y como \textbf{dispositivo} a la GPU o tarjeta gráfica utilizada. \\

La intercomunicación entre \textit{host} y dispositivo sigue un modelo maestro-esclavo. El \textit{host} actúa como maestro y es el encargado de indicar al dispoistivo el código que ha de ejecutar y de mandarlo a la cola del dispositivo. Además, el \textit{host} tiene la posibilidad de trabajar de forma asíncrona con la GPU mientras la cola de trabajos del dispositivo no esté llena. \\

Es de vital importancia a la hora de trabajar con la GPU de tener en cuenta que:\\
\begin{itemize}
    \item a) La GPU tiene muchos más núcleos \textit{(cores)} que una CPU, lo que nos permite realizar mucha más operaciones en el mismo instante. Sin embargo, esto viene a expensas de un menor número de operaciones por segundo de cada núcleo, ya que para distrutar de la cantidad masiva de núcleos que tiene una GPU es necesario que ésta opere a una frecuencia más baja.

    \item b) La GPU tiene su propia estructura de memoria, que ha de usar para poder realizar operaciones. Dentro de la jerarquía de memoria encontramos memoria RAM similar a la que utiliza la CPU a través de la placa base, así como varios niveles de caché. Además, hemos de tener en cuenta que a la hora de ejecutar algo en la GPU vamos a tener un gasto extra de tiempo por el traspaso de información de CPU a GPU y viceversa. Minimizar la información que ha de traspasarse en ambos sentidos así como intentar que toda la información necesaria sea transferida a la vez para sacar máximo potencial del PCI Express y exprimir al máximo posible el uso eficiente de la memoria caché, que en CUDA es habitualmente realizado mediante el manejo de la ``memoria compartida'' es fundamental para obtener mejores resultados, especialmente, aquellos en los que el cuello de botella es la transferencia de datos.

    \item c) Como la GPU tiene su propia memoria dedicada de un tamaño limitado hemos de hacer hincapié en no utilizar soluciones que generan demasiada complejidad espacial, ya que limitan la escalabilidad de los algoritmos.
\end{itemize}

\subsection{Estructura de hebras, bloques y mallas.}
El \textbf{\textit{kernel}} es un fragmento de código especial destino a ser ejecutado en el dispotivo en el que se indica lo que ha de hacer una hebra.\\

Las \textbf{hebras} son la unidad mínima en la arquitectura CUDA. Cada hebra es ejecutada por un núcleo CUDA. Cada hebra es consciente en tiempo de ejecución de su identificador dentro del bloque así como del identificador del bloque en el que se encuentra y el tamaño del mismo, permitiéndonos así repartir el trabajo en función de dichos valores. El \textbf{bloque} se corresponde a un conjunto de hebras que ejecuta el mismo \textit{kernel} y que pueden cooperar entre sí y, al conjunto de esos bloques, se le denomina \textbf{``grid'' o malla}. Tanto las hebras dentro de un bloque como los bloques dentro de una malla puede tener estructuras unidimensionales, bidimensionales y tridimensionales. Las dimensiones de estas estructuras será indicada por el \textit{host} a la hora de ejecutar el \textit{kernel}.\\

CUDA exige que un mínimo de 32 hebras, denominado \textit{warp}, ejecuten instrucciones a la vez, aunque se hagan cálculos innecesarios así como que todas las hebras de un bloque sean ejecutadas por el mismo \textit{Streaming MultiProcessor}, de ahora en adelante, SM, que es uno de los procesadores en el dispositivo y dispone de un número específico de núcleos CUDA, sus propios registros y su propia caché entre otros.\\

Al lanzar un \textit{kernel} hemos de utilizar al menos un bloque de $N$ hebras. Además, en los casos unidimensionales el número de hebras por bloque está limitado a un máximo que depende de la tarjeta gráfica en cuestión.

\subsection{La memoria compartida.}
Dentro de la tarjeta gráfica, nos encontramos con distintos niveles de memoria. Una vez los datos necesarios han sido traspasados del \textit{host} al dispositivo a través del bus PCI Express, esos datos son almacenados en una memoria DRAM de propósito general del dispositivo. Cuando un \textit{kernel} solicita datos de esta memoria, de manera similar a como ocurre en una CPU, los datos solicitados y los colidantes en memoria son colocados a través de varios niveles de caché, que tiene tamaño más limitado que la memoria DRAM pero con un acceso de lectura y escritura mucho más rápido.\\

La \textbf{memoria compartida} es una abstracción para una región especial de la caché asociada a un bloque que es explícitamente usada por el programador en el \textit{kernel}, agilizando así considerablemente las transferencias de memoria en el dispositivo. En el cuadro \ref{tab:cudamemory}, podemos ver un resumen de los tipos de memoria existentes, dónde se pueden usar y dónde se encuentran dichos datos en el dispositivo.

\begin{table}[ht]
\begin{tabular}{|c|c|c|c|}
\hline
\textbf{Memoria}    & \textbf{Localización}                                           & \textbf{\begin{tabular}[c]{@{}c@{}}Acceso\\ (E = Escribir)\\ (L = Leer)\end{tabular}} & \textbf{\begin{tabular}[c]{@{}c@{}}Existente\\ hasta fin\\ de\end{tabular}} \\ \hline
\textbf{Registro}   & Caché                                                           & Kernel (E/L)                                                                          & Hebra                                                                       \\ \hline
\textbf{Local}      & \begin{tabular}[c]{@{}c@{}}DRAM\\ (Caché tras uso)\end{tabular} & Kernel (E/L)                                                                          & Hebra                                                                       \\ \hline
\textbf{Compartida} & Caché                                                           & Kernel (E/L)                                                                          & Bloque                                                                      \\ \hline
\textbf{Global}     & \begin{tabular}[c]{@{}c@{}}DRAM\\ (Caché tras uso)\end{tabular} & \begin{tabular}[c]{@{}c@{}}Host (E/L)\\ Kernel (E/L)\end{tabular}                     & \begin{tabular}[c]{@{}c@{}}Aplicación\\ o uso de free\end{tabular}          \\ \hline
\textbf{Constante}  & \begin{tabular}[c]{@{}c@{}}DRAM\\ (Caché tras uso)\end{tabular} & \begin{tabular}[c]{@{}c@{}}Host (E/L)\\ Kernel (L)\end{tabular}                       & \begin{tabular}[c]{@{}c@{}}Aplicación \\ o uso de free\end{tabular}         \\ \hline
\end{tabular}
\caption{Resumen de los tipos de memoria en CUDA.}
\label{tab:cudamemory}
\end{table}

\subsection{Python: Numba y CuPy.}
Para desarrollar el código asociado a este proyecto, hemos optado por utilizar \textbf{Python} en vez de los tradicionales C o C++. El uso de \textit{Python} nos permite un desarrollo de los algoritmos más rápido así como el acceso a abstracciones de más alto nivel mediante el uso de la librerías \textbf{\textit{Numba}} y \textbf{\textit{CuPy}},  así como una mayor facilidad para la distribución del código, si se desea, mediante el uso de \textit{PyPI(Python Package Index)}, el repositorio de paquetes para Python. \\

\textbf{Numba} \cite{numba} es un paquete para Python cuyo objetivo es la aceleración compilado fragmentos de código utilizando el compilador LLVM y dando la oportunidad de paralelizar código tanto para la CPU como para la GPU. En concreto, para las GPUs CUDA, proporciona al usuario un subconjunto de las características de CUDA con un nivel de abstracción mayor. Con eso no sólo conseguimos poder trabajar con CUDA desde Python sino, también evitar, si lo deseamos, manejar los traspasos de memoria entre host y dispositivo o la necesidad de indicar todos los tipos a la hora de inicializar un \textit{kernel} entre otras ventajas.
\begin{code}
\begin{minted}[fontsize=\footnotesize]{python}
from numba import cuda
import numpy as np
# Definimos el kernel
@cuda.jit
def aumentar_en_1(un_array):
  # Cogemos el índice de la hebra
    pos = cuda.grid(1)

    # Si el índice está en el rango del array
    # incrementamos su valor
    if pos < un_array.size:
        un_array[pos] += 1

if __name__ == '__main__':
  # Declaramos un array de 10000 ceros
  ejemplo = np.zeros(10000)
  # Calculamos el número de bloques necesario
  bloques = ejemplo.size // 128 + 1
  # Lanzamos el kernel con bloques de 128 hebras
  aumentar_en_1[bloques, 128](ejemplo)
\end{minted}
\captionof{listing}{Kernel para incrementar en 1 los elementos de un array.\\\\}
\label{code:numbaexample}
\end{code}

\textbf{CuPy} \cite{cupy} es otro paquete de Python que, por un lado y de manera similar a Numba, nos permite generar kernels para CUDA en este caso de manera similar a los de C/C++ así como facilidades para generar kernels en los que se implementa reducciones u operaciones elemento a elemento en un array. Por otro lado, proporciona una API similar a la de NumPy pero las operaciones están implementadas utilizando CUDA. Además, CuPy está implementado de manera que permite utilizar directamente sus estructuras de datos sobre kernels de Numba, lo que nos permite combinar elementos de ambos paquetes según nos interese.

\subsection{Spark.}
\textit{Apache Spark} es un \textit{framework} de código abierto y propósito general para sistemas distribuidos de computación en clúster que proporciona una API utilizable desde los lenguajes de programación en Scala, Java, Python y R. El \textit{framework} fundamenta su arquitectura en el \textit{RDD (Resilient Distributed DataSet)}, que es una estructura de datos de sólo lectura distribuida en un clúster de máquinas, mantenida durante toda la computación y con tolerancia a fallos. Además, proporciona otras herramientas de alto nivel como ML/MLib, una librería con algoritmos de \textit{machine learning}.\\

Utilizando la API de Python, podemos combinar el uso de \textit{Spark} y \textit{Numba CUDA} para afrontar problemas de grandes dimensiones, ya que el \textit{RDD} nos permite trabajar con subconjuntos de esos datos posibilitando incluso llevar las implementaciones realizadas a un clúster con múltiples sistemas con dispositivos GPU \textit{CUDA} cont todas las dependencias necesarias instaladas. \\

La distribución de trabajo en Spark se realizará utilizando la transformación \textit{mapPartitions} del \textit{RDD} de \textit{Spark}, que generá un nuevo RDD a partir de los resultados obtenidos al aplicar la función pasada a \textit{mapPartitions} como parámetro a cada una de las funciones.

