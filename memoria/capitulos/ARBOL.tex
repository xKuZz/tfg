\chapter{Árboles de decisión.}

Un árbol de decisión es un modelo de aprendizaje automático no lineal que permite resolver los problemas de manera jerárquica. Una vez el árbol es construido el árbol consta de dos tipos de nodos: los nodos de decisión y los nodos terminales. Los \textbf{nodos de decisión}, son reglas que nos idican cómo hemos de navegar el árbol para clasificar una muestra. Lo habitual es que dicho nodo esté asociado a una carácterística del problema y un valor de tal manera que si la condición se cumple, tomamos un camino y, en caso contrario, tomamos el otro. Una vez alcanzamos un \textbf{nodo terminal} (correspondientes a las hojas de los árboles), conocemos la categoría a la que corresponde una muestra.\\\\

La mayoría de los algoritmos secuenciales utilizan metodologías de tipo voraz \textit{(greedy)} para la generación de manera recursiva de la estructura del árbol. Dentro de esta categoría encontramos los algoritmos más conocidos como ID3, C4.5 y CART.
